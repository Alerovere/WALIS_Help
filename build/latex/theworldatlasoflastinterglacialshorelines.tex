%% Generated by Sphinx.
\def\sphinxdocclass{report}
\documentclass[letterpaper,10pt,english]{sphinxmanual}
\ifdefined\pdfpxdimen
   \let\sphinxpxdimen\pdfpxdimen\else\newdimen\sphinxpxdimen
\fi \sphinxpxdimen=.75bp\relax

\PassOptionsToPackage{warn}{textcomp}
\usepackage[utf8]{inputenc}
\ifdefined\DeclareUnicodeCharacter
% support both utf8 and utf8x syntaxes
  \ifdefined\DeclareUnicodeCharacterAsOptional
    \def\sphinxDUC#1{\DeclareUnicodeCharacter{"#1}}
  \else
    \let\sphinxDUC\DeclareUnicodeCharacter
  \fi
  \sphinxDUC{00A0}{\nobreakspace}
  \sphinxDUC{2500}{\sphinxunichar{2500}}
  \sphinxDUC{2502}{\sphinxunichar{2502}}
  \sphinxDUC{2514}{\sphinxunichar{2514}}
  \sphinxDUC{251C}{\sphinxunichar{251C}}
  \sphinxDUC{2572}{\textbackslash}
\fi
\usepackage{cmap}
\usepackage[T1]{fontenc}
\usepackage{amsmath,amssymb,amstext}
\usepackage{babel}



\usepackage{times}
\expandafter\ifx\csname T@LGR\endcsname\relax
\else
% LGR was declared as font encoding
  \substitutefont{LGR}{\rmdefault}{cmr}
  \substitutefont{LGR}{\sfdefault}{cmss}
  \substitutefont{LGR}{\ttdefault}{cmtt}
\fi
\expandafter\ifx\csname T@X2\endcsname\relax
  \expandafter\ifx\csname T@T2A\endcsname\relax
  \else
  % T2A was declared as font encoding
    \substitutefont{T2A}{\rmdefault}{cmr}
    \substitutefont{T2A}{\sfdefault}{cmss}
    \substitutefont{T2A}{\ttdefault}{cmtt}
  \fi
\else
% X2 was declared as font encoding
  \substitutefont{X2}{\rmdefault}{cmr}
  \substitutefont{X2}{\sfdefault}{cmss}
  \substitutefont{X2}{\ttdefault}{cmtt}
\fi


\usepackage[Bjarne]{fncychap}
\usepackage{sphinx}

\fvset{fontsize=\small}
\usepackage{geometry}


% Include hyperref last.
\usepackage{hyperref}
% Fix anchor placement for figures with captions.
\usepackage{hypcap}% it must be loaded after hyperref.
% Set up styles of URL: it should be placed after hyperref.
\urlstyle{same}

\addto\captionsenglish{\renewcommand{\contentsname}{Contents:}}

\usepackage{sphinxmessages}
\setcounter{tocdepth}{1}



\title{The World Atlas of Last Interglacial Shorelines}
\date{Mar 03, 2021}
\release{2.0}
\author{A.\@{} Rovere (editor)}
\newcommand{\sphinxlogo}{\vbox{}}
\renewcommand{\releasename}{Release}
\makeindex
\begin{document}

\pagestyle{empty}
\sphinxmaketitle
\pagestyle{plain}
\sphinxtableofcontents
\pagestyle{normal}
\phantomsection\label{\detokenize{index::doc}}



\chapter{Introduction}
\label{\detokenize{first:introduction}}\label{\detokenize{first::doc}}

\section{Overview}
\label{\detokenize{first:overview}}
The World Atlas of Last Interglacial Shorelines (WALIS) is a database containing information on published stratigraphic data (and dated samples) indicating the position of sea level during Marine Isotopic Stage 5. WALIS is composed by different elements: 1) the mySQL database, containing the data tables and associated relations; 2) a database interface based on PhP; 3) a series of Jupyter notebooks that download and process data.

\begin{sphinxadmonition}{note}{Note:}
As of March 2021, the Jupyter notebooks are still in preparation, and not publicly available
\end{sphinxadmonition}

The data insersion interface is available at \sphinxhref{https://warmcoasts.eu/world-atlas.html}{this page}. Geoscientists wishing to contribute data must register to our system. This will allow them to use an intuitive interface to insert both published and new sea level indicators, ages and metadata. During the compilation process, data remain private and are accessible only by the registered user. This is done to allow registered users to keep inserting and modifying data points until they are ready for publication. Registration to the system is (and will always be) free, but users must read and accept the WALIS data \sphinxhref{https://warmcoasts.eu/privacy.html}{policy}.

WALIS aims to make Last Interglacial sea level data open access and readily available. Within the database interface, a tool allows to export the data inserted by the logged user as a multi\sphinxhyphen{}sheet xls file. Data creators are encouraged, once their work is finalized, to submit this file to a repository where it will get a DOI. To this purpose, we set up a \sphinxhref{https://zenodo.org/communities/walis\_database/}{Zenodo Repository}. As a user, you are free to decide to submit to another open\sphinxhyphen{}access repository, but if you do please let us know by writing a message to \sphinxhref{mailto:arovere@marum.de}{arovere@marum.de}.

Once the data inserted by a data creator has been assigned a DOI from an open\sphinxhyphen{}access repository, it can be downloaded and re\sphinxhyphen{}used freely (complying with the simple rules at the bottom of this page). Periodically, WALIS data will be collated into a single release, containing all the data that were assigned a DOI up to the release date. The release will contain WALIS data in different formats, as well as scripts to query the database. The first release is planned to coincide with the closing of an ongoing Special Issue in the journal Earth System Science Data.

\begin{sphinxadmonition}{note}{Note:}
The first release of the database is foreseen by the end of 2021
\end{sphinxadmonition}

This document contains help tips to compile data using the interface and fully documents the database structure, including field types and table connections.

\begin{sphinxadmonition}{tip}{Tip:}
For more information, see the atlas webpage: \sphinxurl{https://warmcoasts.eu/world-atlas.html}.
\end{sphinxadmonition}


\section{How to cite data}
\label{\detokenize{first:how-to-cite-data}}
If you use the database, we kindly ask you to provide credit where it is due following our citation guidelines.

\begin{sphinxadmonition}{warning}{Warning:}
WALIS contains data that were originally published in research articles or technical reports. These data were standardized by data contributors, who might be different from the original authors. When exporting the database from the WALIS interface, a suggested acknowledgment is generated alongside with a suggested reference list. We strongly suggest not to remove these elements.
\end{sphinxadmonition}

If you use WALIS, we kindly ask you to follow these simple rules to acknowledge those who worked on it:

\sphinxstylestrong{1) Cite the original authors} \sphinxhyphen{} Please maintain the original citations of each datapoint. If you think that some datapoint needs further attribution, please inform us (see contact page).

\sphinxstylestrong{2) Acknowledge the database contributor} \sphinxhyphen{} A database contributor is a scientist or a group of scientists who standardized and in some instance re\sphinxhyphen{}interpreted the original research papers where the data was published. For each datapoint in WALIS, we provide the name of the first data creator and the last scientist who edited the data.

\sphinxstylestrong{3) Acknowledge the creators of the database and interface} \sphinxhyphen{} The database template used in this study was developed by the ERC Starting Grant “Warmcoasts” (ERC\sphinxhyphen{}StG\sphinxhyphen{}802414) and is a community effort under the PALSEA (PAGES / INQUA) working group.

\begin{sphinxadmonition}{tip}{Tip:}
\sphinxstylestrong{Suggested acknowledgment:} The data used in this study were {[}\sphinxstyleemphasis{extracted from / compiled in}{]} WALIS, a sea\sphinxhyphen{}level database interface developed by the ERC Starting Grant “WARMCOASTS” (ERC\sphinxhyphen{}StG\sphinxhyphen{}802414), in collaboration with PALSEA (PAGES / INQUA) working group. The database structure was designed by A. Rovere, D. Ryan, T. Lorscheid, A. Dutton, P. Chutcharavan, D. Brill, N. Jankowski, D. Mueller, M. Bartz, E. Gowan and K. Cohen. The data points used in this study were contributed to WALIS by {[}\sphinxstyleemphasis{list names of contributors here}{]}.
\end{sphinxadmonition}


\section{How to cite this document}
\label{\detokenize{first:how-to-cite-this-document}}
We suggest to cite this document whenever referring to the database. This work can be cited as:

Rovere, Alessio, Ryan, Deirdre, Murray\sphinxhyphen{}Wallace, Colin, Simms, Alexander, Vacchi, Matteo, Dutton, Andrea, Lorscheid, Thomas, Chutcharavan, Peter, Brill, Dominik, Bartz, Melanie, Jankowski, Nathan, Mueller, Daniela, Cohen, Kim, Gowan, Evan, 2020. Descriptions of database fields for the World Atlas of Last Interglacial Shorelines (WALIS). doi:10.5281/zenodo.3961544

\sphinxstylestrong{Bibtex code}

\begin{sphinxVerbatim}[commandchars=\\\{\}]
\PYG{n+nc}{@misc}\PYG{p}{\PYGZob{}}\PYG{n+nl}{rovere\PYGZus{}alessio\PYGZus{}2020\PYGZus{}3961544}\PYG{p}{,}
\PYG{n+na}{author}       \PYG{p}{=} \PYG{l+s}{\PYGZob{}}\PYG{l+s}{Rovere, Alessio and}
\PYG{l+s}{                                Ryan, Deirdre and}
\PYG{l+s}{                                Murray\PYGZhy{}Wallace, Colin and}
\PYG{l+s}{                                Simms, Alexander and}
\PYG{l+s}{                                Vacchi, Matteo and}
\PYG{l+s}{                                Dutton, Andrea and}
\PYG{l+s}{                                Lorscheid, Thomas and}
\PYG{l+s}{                                Chutcharavan, Peter and}
\PYG{l+s}{                                Brill, Dominik and}
\PYG{l+s}{                                Bartz, Melanie and}
\PYG{l+s}{                                Jankowski, Nathan and}
\PYG{l+s}{                                Mueller, Daniela and}
\PYG{l+s}{                                Cohen, Kim and}
\PYG{l+s}{                                Gowan, Evan}\PYG{l+s}{\PYGZcb{}}\PYG{p}{,}
\PYG{n+na}{title}        \PYG{p}{=} \PYG{l+s}{\PYGZob{}}\PYG{l+s}{\PYGZob{}}\PYG{l+s}{Descriptions of database fields for the World}
\PYG{l+s}{                                Atlas of Last Interglacial Shorelines (WALIS)}\PYG{l+s}{\PYGZcb{}}\PYG{l+s}{\PYGZcb{}}\PYG{p}{,}
\PYG{n+na}{year}         \PYG{p}{=} \PYG{l+m}{2020}\PYG{p}{,}
\PYG{n+na}{publisher}    \PYG{p}{=} \PYG{l+s}{\PYGZob{}}\PYG{l+s}{Zenodo}\PYG{l+s}{\PYGZcb{}}\PYG{p}{,}
\PYG{n+na}{version}      \PYG{p}{=} \PYG{l+s}{\PYGZob{}}\PYG{l+s}{1,0}\PYG{l+s}{\PYGZcb{}}\PYG{p}{,}
\PYG{n+na}{doi}          \PYG{p}{=} \PYG{l+s}{\PYGZob{}}\PYG{l+s}{10.5281/zenodo.3961544}\PYG{l+s}{\PYGZcb{}}\PYG{p}{,}
\PYG{n+na}{url}          \PYG{p}{=} \PYG{l+s}{\PYGZob{}}\PYG{l+s}{https://doi.org/10.5281/zenodo.3961544}\PYG{l+s}{\PYGZcb{}}\PYG{p}{\PYGZcb{}}
\end{sphinxVerbatim}


\section{Funding}
\label{\detokenize{first:funding}}
WALIS \sphinxhyphen{} The World Atlas of Last Interglacial Shorelines is an online database collecting data and metadata on paleo sea level proxies. WALIS was built as part of the ERC Starting Grant “Warmcoasts” (ERC\sphinxhyphen{}StG\sphinxhyphen{}802414)
\begin{quote}

\noindent{\hspace*{\fill}\sphinxincludegraphics[width=400\sphinxpxdimen]{{WALIS_logos}.jpg}\hspace*{\fill}}
\end{quote}


\section{License}
\label{\detokenize{first:license}}
This work is licensed under a \sphinxhref{http://creativecommons.org/licenses/by-sa/4.0/}{Creative Commons Attribution\sphinxhyphen{}ShareAlike 4.0 International License}, \sphinxstylestrong{unless specified otherwise}.
\begin{quote}

\noindent{\hspace*{\fill}\sphinxincludegraphics[width=100\sphinxpxdimen]{{88x31}.png}\hspace*{\fill}}
\end{quote}


\chapter{Technical data}
\label{\detokenize{tech_data:technical-data}}\label{\detokenize{tech_data::doc}}
WALIS is composed by a database, a user interface and a series of Jupyter notebooks that allow extracting and processing the database records. These three elements are briefly introduced below.


\section{Database}
\label{\detokenize{tech_data:database}}
WALIS data is contained in a mySQL (5.7) database, hosted on an Amazon Web Services instance. The database is managed offline with \sphinxhref{https://www.heidisql.com/}{HeidiSQL}. and \sphinxhref{https://www.mysql.com/products/workbench/}{MySQL Workbench}. The database contains 23 tables and 1 view. Seven of the 23 tables are used for user and groups management within the interface, while the remaining 16 tables and 1 view are dedicated to sea\sphinxhyphen{}level data and meta\sphinxhyphen{}data.

\begin{sphinxadmonition}{warning}{Warning:}
Only one\sphinxhyphen{}to\sphinxhyphen{}one relationships are defined in the database directly. One\sphinxhyphen{}to\sphinxhyphen{}many and many\sphinxhyphen{}to\sphinxhyphen{}many relationships are defined via comma\sphinxhyphen{}separated fields and managed in the PhP interface directly.
\end{sphinxadmonition}


\section{Interface}
\label{\detokenize{tech_data:interface}}
The WALIS interface is written in PhP (7.0). It allows users to register and insert data. The bulk of the interface was done with \sphinxhref{https://www.scriptcase.net}{Scriptcase}, a commercial PhP Web Development tool. Within the interface, there is the possibility to download a multi\sphinxhyphen{}sheet Excel file containing all data inserted by the logged user. This tool is coded in Python, served via a \sphinxhref{https://flask.palletsprojects.com/en/1.1.x/}{Flask} web app.

\begin{sphinxadmonition}{note}{Note:}
A modification of the Flask web app is in preparation, to allow the user to export a zip file containing not only Excel file (a proprietary format), but also a series of csv and json files.
\end{sphinxadmonition}


\section{Jupyter notebooks}
\label{\detokenize{tech_data:jupyter-notebooks}}
A series of Jupyter notebooks allows extracting the data and making simple maps/statistics.

\begin{sphinxadmonition}{note}{Note:}
As of January 2021, the Jupyter notebooks are still in preparation, and not publicly available
\end{sphinxadmonition}


\chapter{Database tables}
\label{\detokenize{database:database-tables}}\label{\detokenize{database::doc}}
In the following section, each database table contained in WALIS is described. Six fields are given:
\begin{enumerate}
\sphinxsetlistlabels{\arabic}{enumi}{enumii}{}{.}%
\item {} 
Field: Column name as it appears in the database

\item {} 
Type: Type of data accepted by the mySQL database

\item {} 
Collation: Econding

\item {} 
Null: are null values are allowed?

\item {} 
Key: primary (PRI) or foreign (MUL) key

\item {} 
Comment: label within the PhP interface

\end{enumerate}


\section{Table: rsl}
\label{\detokenize{database:table-rsl}}
This table contains data and metadata related to paleo relative sea level datapoints from stratigraphic or geomorphic sea level indicators.


\begin{savenotes}\sphinxatlongtablestart\begin{longtable}[c]{|\X{20}{105}|\X{15}{105}|\X{25}{105}|\X{15}{105}|\X{10}{105}|\X{20}{105}|}
\hline
\sphinxstyletheadfamily 
Field
&\sphinxstyletheadfamily 
Type
&\sphinxstyletheadfamily 
Collation
&\sphinxstyletheadfamily 
Null
&\sphinxstyletheadfamily 
Key
&\sphinxstyletheadfamily 
Comment
\\
\hline
\endfirsthead

\multicolumn{6}{c}%
{\makebox[0pt]{\sphinxtablecontinued{\tablename\ \thetable{} \textendash{} continued from previous page}}}\\
\hline
\sphinxstyletheadfamily 
Field
&\sphinxstyletheadfamily 
Type
&\sphinxstyletheadfamily 
Collation
&\sphinxstyletheadfamily 
Null
&\sphinxstyletheadfamily 
Key
&\sphinxstyletheadfamily 
Comment
\\
\hline
\endhead

\hline
\multicolumn{6}{r}{\makebox[0pt][r]{\sphinxtablecontinued{continues on next page}}}\\
\endfoot

\endlastfoot

RSL \_ ID
&
int(11)
&&
NO
&
PRI
&
WALIS RSL ID
\\
\hline
Public
&
int(11)
&&
NO
&&
Is this datapoint public?
\\
\hline
Site
&
varchar(40)
&
utf8 \_ general \_ ci
&
NO
&&
Site
\\
\hline
AddLocInfo
&
varchar(100)
&
utf8 \_ general \_ ci
&
YES
&&
Subsite
\\
\hline
Nation
&
int(11)
&&
NO
&&
Nation
\\
\hline
Region
&
int(11)
&&
NO
&&
Region
\\
\hline
Ref
&
varchar(255)
&
utf8 \_ general \_ ci
&
NO
&
MUL
&
Main reference
\\
\hline
addRef
&
varchar(45)
&
utf8 \_ general \_ ci
&
YES
&&
Additional references
\\
\hline
RSLlat
&
decimal(10,6)
&&
NO
&&
Latitude (decimal degrees)
\\
\hline
RSLlon
&
decimal(10,6)
&&
NO
&&
Longitude (decimal degrees)
\\
\hline
HrzPosTech
&
int(11)
&&
YES
&
MUL
&
Horizontal Positioning Technique
\\
\hline
Limiting
&
varchar(45)
&
utf8 \_ general \_ ci
&
NO
&&
Is this a marine/terrestrial limiting record?
\\
\hline
Indicator
&
int(11)
&&
YES
&
MUL
&
Type of RSL Indicator
\\
\hline
Ind \_ descr \_ det
&
longtext
&
utf8 \_ general \_ ci
&
YES
&&
Indicator description
\\
\hline
ULmodAn
&
decimal(10,2)
&&
YES
&&
Upper limit of modern analog (m)
\\
\hline
LLmodAn
&
decimal(10,2)
&&
YES
&&
Lower limit of modern analog (m)
\\
\hline
Quant \_ ind \_ mean
&
varchar(45)
&
utf8 \_ general \_ ci
&
YES
&&
Quantification of indicative meaning
\\
\hline
SLdatum
&
int(11)
&&
NO
&&
Sea level datum
\\
\hline
VrtMeasTech
&
int(11)
&&
NO
&&
Elevation measurement technique
\\
\hline
Upper \_ Lower \_ elev
&
varchar(45)
&
utf8 \_ general \_ ci
&
YES
&&
Do you want to insert upper and lower elevation limits?
\\
\hline
Elev \_ upper
&
varchar(45)
&
utf8 \_ general \_ ci
&
YES
&&
Upper elevation of indicator (m)
\\
\hline
Elev \_ lower
&
varchar(45)
&
utf8 \_ general \_ ci
&
YES
&&
Lower elevation of indicator (m)
\\
\hline
UppLowErr
&
varchar(45)
&
utf8 \_ general \_ ci
&
YES
&&
Upper/Lower elevation measurement error (1\sphinxhyphen{}sigma) (m)
\\
\hline
RSLelev
&
decimal(10,2)
&&
YES
&&
RSL indicator elevation (m)
\\
\hline
RSLelevErr
&
decimal(10,2)
&&
YES
&&
RSL indicator elevation error (m)
\\
\hline
ElevationNotes
&
longtext
&
utf8 \_ general \_ ci
&
YES
&&
Notes on elevation and indicative range
\\
\hline
RWL
&
decimal(10,2)
&&
YES
&&
Reference Water Level (m)
\\
\hline
IR
&
decimal(10,2)
&&
YES
&&
Indicative Range (m)
\\
\hline
PaleoRSL
&
decimal(10,2)
&&
YES
&&
Paleo Relative Sea Level (m)
\\
\hline
PaleoRSLunc
&
decimal(10,2)
&&
YES
&&
Paleo Relative Sea Level Uncertainty (m)
\\
\hline
ChoiceVLM
&
varchar(45)
&
utf8 \_ general \_ ci
&
YES
&&
Is data on vertical land movements (independent from the sea level record) available?
\\
\hline
Tect \_ cat
&
varchar(45)
&
utf8 \_ general \_ ci
&
YES
&&
Tectonic category
\\
\hline
tect \_ cat \_ comments
&
longtext
&
utf8 \_ general \_ ci
&
YES
&&
Comments on tectonic category
\\
\hline
Publ \_ VLM
&
varchar(45)
&
utf8 \_ general \_ ci
&
YES
&&
Published VLM rate (m/ky)
\\
\hline
Publ \_ VLM \_ unc
&
varchar(45)
&
utf8 \_ general \_ ci
&
YES
&&
Published VLM rate uncertainty (m/ky)
\\
\hline
Interpr \_ VLM
&
varchar(45)
&
utf8 \_ general \_ ci
&
YES
&&
Interpreted VLM rate (m/ky)
\\
\hline
Interpr \_ VLM \_ unc
&
varchar(45)
&
utf8 \_ general \_ ci
&
YES
&&
Interpreted VLM rate (m/ky) uncertainty
\\
\hline
Comments \_ VLM
&
longtext
&
utf8 \_ general \_ ci
&
YES
&&
Comments on VLM rates
\\
\hline
SelectAge
&
varchar(45)
&
utf8 \_ general \_ ci
&
NO
&&
Age attribution
\\
\hline
Useries \_ time
&
varchar(45)
&
utf8 \_ general \_ ci
&
YES
&&
U\sphinxhyphen{}Series constraint
\\
\hline
Useries
&
mediumtext
&
utf8 \_ general \_ ci
&
YES
&&
U\sphinxhyphen{}series age IDs
\\
\hline
AAR \_ time
&
varchar(45)
&
utf8 \_ general \_ ci
&
YES
&&
AAR constraint
\\
\hline
AAR
&
mediumtext
&
utf8 \_ general \_ ci
&
YES
&&
Amino Acid Racemization age IDs
\\
\hline
ESR \_ time
&
varchar(45)
&
utf8 \_ general \_ ci
&
YES
&&
ESR constraint
\\
\hline
ESR
&
mediumtext
&
utf8 \_ general \_ ci
&
YES
&&
Electro Spin Resonance age IDs
\\
\hline
LUM \_ time
&
varchar(45)
&
utf8 \_ general \_ ci
&
YES
&&
Luminescence constraint
\\
\hline
Luminescence
&
mediumtext
&
utf8 \_ general \_ ci
&
YES
&&
Luminescence age IDs
\\
\hline
Strat \_ time
&
varchar(45)
&
utf8 \_ general \_ ci
&
YES
&&
Stratigraphic constraint
\\
\hline
Stratcontext
&
mediumtext
&
utf8 \_ general \_ ci
&
YES
&&
Stratigraphic context/age IDs
\\
\hline
Other \_ time
&
varchar(45)
&
utf8 \_ general \_ ci
&
YES
&&
Other age constraint
\\
\hline
Other \_ age
&
mediumtext
&
utf8 \_ general \_ ci
&
YES
&&
Other age constraints IDs
\\
\hline
Qual
&
int(11)
&&
NO
&&
Quality of RSL data
\\
\hline
Qual \_ Age
&
int(11)
&&
NO
&&
Quality of age information
\\
\hline
Qualnotes
&
longtext
&
utf8 \_ general \_ ci
&
YES
&&
Quality notes
\\
\hline
Createdby
&
varchar(45)
&
utf8 \_ general \_ ci
&
YES
&&
Record created by
\\
\hline
Updatedby
&
varchar(45)
&
utf8 \_ general \_ ci
&
YES
&&
Record updated by
\\
\hline
LastUpdate
&
datetime
&&
YES
&&
Last Update
\\
\hline
coord
&
point
&&
YES
&&
MySQL coordinates
\\
\hline
\end{longtable}\sphinxatlongtableend\end{savenotes}


\section{Table: vertmeastech}
\label{\detokenize{database:table-vertmeastech}}
This table contains elevation measurement techniques used in WALIS.


\begin{savenotes}\sphinxattablestart
\centering
\begin{tabular}[t]{|\X{20}{105}|\X{15}{105}|\X{25}{105}|\X{15}{105}|\X{10}{105}|\X{20}{105}|}
\hline
\sphinxstyletheadfamily 
Field
&\sphinxstyletheadfamily 
Type
&\sphinxstyletheadfamily 
Collation
&\sphinxstyletheadfamily 
Null
&\sphinxstyletheadfamily 
Key
&\sphinxstyletheadfamily 
Comment
\\
\hline
idvrtpostech
&
int(11)
&&
NO
&
PRI
&
WALIS ElevMeas ID
\\
\hline
VrtType
&
mediumtext
&
utf8 \_ general \_ ci
&
YES
&&
Measurement technique
\\
\hline
VrtDescr
&
longtext
&
utf8 \_ general \_ ci
&
YES
&&
Description
\\
\hline
VrtAcc
&
mediumtext
&
utf8 \_ general \_ ci
&
YES
&&
Typical accuracy
\\
\hline
Createdby
&
varchar(45)
&
utf8 \_ general \_ ci
&
YES
&&
Record created by
\\
\hline
Updatedby
&
varchar(45)
&
utf8 \_ general \_ ci
&
YES
&&
Record updated by
\\
\hline
LastUpdate
&
datetime
&&
YES
&&
Last Update
\\
\hline
\end{tabular}
\par
\sphinxattableend\end{savenotes}


\section{Table: hrzmeastech}
\label{\detokenize{database:table-hrzmeastech}}
This table contains geographic positioning techniques used in WALIS.


\begin{savenotes}\sphinxattablestart
\centering
\begin{tabular}[t]{|\X{20}{105}|\X{15}{105}|\X{25}{105}|\X{15}{105}|\X{10}{105}|\X{20}{105}|}
\hline
\sphinxstyletheadfamily 
Field
&\sphinxstyletheadfamily 
Type
&\sphinxstyletheadfamily 
Collation
&\sphinxstyletheadfamily 
Null
&\sphinxstyletheadfamily 
Key
&\sphinxstyletheadfamily 
Comment
\\
\hline
idposhrz
&
int(11)
&&
NO
&
PRI
&
WALIS GeoPos ID
\\
\hline
HrzType
&
mediumtext
&
utf8 \_ general \_ ci
&
YES
&&
Measurement technique
\\
\hline
HrzDescr
&
longtext
&
utf8 \_ general \_ ci
&
YES
&&
Description
\\
\hline
HrzAcc
&
longtext
&
utf8 \_ general \_ ci
&
YES
&&
Typical accuracy
\\
\hline
Createdby
&
text
&
utf8 \_ general \_ ci
&
YES
&&
Record created by
\\
\hline
Updatedby
&
text
&
utf8 \_ general \_ ci
&
YES
&&
Record updated by
\\
\hline
LastUpdate
&
datetime
&&
YES
&&
Last Update
\\
\hline
\end{tabular}
\par
\sphinxattableend\end{savenotes}


\section{Table: countries}
\label{\detokenize{database:table-countries}}
This table contains the countries selectable in the interface.


\begin{savenotes}\sphinxattablestart
\centering
\begin{tabular}[t]{|\X{20}{105}|\X{15}{105}|\X{25}{105}|\X{15}{105}|\X{10}{105}|\X{20}{105}|}
\hline
\sphinxstyletheadfamily 
Field
&\sphinxstyletheadfamily 
Type
&\sphinxstyletheadfamily 
Collation
&\sphinxstyletheadfamily 
Null
&\sphinxstyletheadfamily 
Key
&\sphinxstyletheadfamily 
Comment
\\
\hline
id
&
smallint(5) unsigned
&&
NO
&
PRI
&
Country ID
\\
\hline
name
&
varchar(255)
&
latin1 \_ swedish \_ ci
&
NO
&
UNI
&
Country name
\\
\hline
\end{tabular}
\par
\sphinxattableend\end{savenotes}


\section{Table: regions}
\label{\detokenize{database:table-regions}}
This table contains the administrative regions selectable in the interface.


\begin{savenotes}\sphinxattablestart
\centering
\begin{tabular}[t]{|\X{20}{105}|\X{15}{105}|\X{25}{105}|\X{15}{105}|\X{10}{105}|\X{20}{105}|}
\hline
\sphinxstyletheadfamily 
Field
&\sphinxstyletheadfamily 
Type
&\sphinxstyletheadfamily 
Collation
&\sphinxstyletheadfamily 
Null
&\sphinxstyletheadfamily 
Key
&\sphinxstyletheadfamily 
Comment
\\
\hline
id
&
int(11) unsigned
&&
NO
&
PRI
&
Region ID
\\
\hline
name
&
varchar(255)
&
utf8 \_ general \_ ci
&
NO
&&
Region name
\\
\hline
country \_ id
&
smallint(5) unsigned
&&
NO
&
MUL
&
Country ID
\\
\hline
\end{tabular}
\par
\sphinxattableend\end{savenotes}


\section{Table: MIS\_ages}
\label{\detokenize{database:table-mis-ages}}
This table contains the Marine Isotopic Stages selectable in the interface.


\begin{savenotes}\sphinxattablestart
\centering
\begin{tabular}[t]{|\X{20}{105}|\X{15}{105}|\X{25}{105}|\X{15}{105}|\X{10}{105}|\X{20}{105}|}
\hline
\sphinxstyletheadfamily 
Field
&\sphinxstyletheadfamily 
Type
&\sphinxstyletheadfamily 
Collation
&\sphinxstyletheadfamily 
Null
&\sphinxstyletheadfamily 
Key
&\sphinxstyletheadfamily 
Comment
\\
\hline
MIS ID
&
int(11)
&&
NO
&
PRI
&
WALIS MIS ID
\\
\hline
MIS name
&
varchar(45)
&
utf8 \_ general \_ ci
&
YES
&&
Name of Marine Isotopic stage
\\
\hline
MIS peak age
&
varchar(45)
&
utf8 \_ general \_ ci
&
YES
&&
MIS peak age (ka)
\\
\hline
MIS start age
&
varchar(45)
&
utf8 \_ general \_ ci
&
YES
&&
MIS start age (ka)
\\
\hline
MIS end age
&
varchar(45)
&
utf8 \_ general \_ ci
&
YES
&&
MIS end age (ka)
\\
\hline
\end{tabular}
\par
\sphinxattableend\end{savenotes}


\section{Table: rsl\_ind}
\label{\detokenize{database:table-rsl-ind}}
This table contains the types of RSL indicators inserted in WALIS.


\begin{savenotes}\sphinxattablestart
\centering
\begin{tabular}[t]{|\X{20}{105}|\X{15}{105}|\X{25}{105}|\X{15}{105}|\X{10}{105}|\X{20}{105}|}
\hline
\sphinxstyletheadfamily 
Field
&\sphinxstyletheadfamily 
Type
&\sphinxstyletheadfamily 
Collation
&\sphinxstyletheadfamily 
Null
&\sphinxstyletheadfamily 
Key
&\sphinxstyletheadfamily 
Comment
\\
\hline
idrsl \_ ind
&
int(11)
&&
NO
&
PRI
&
WALIS RSLind ID
\\
\hline
Ind \_ name
&
mediumtext
&
utf8 \_ general \_ ci
&
YES
&&
Name of RSL indicator
\\
\hline
Ind \_ descr
&
longtext
&
utf8 \_ general \_ ci
&
YES
&&
Description of RSL indicator
\\
\hline
RWL \_ descr
&
longtext
&
utf8 \_ general \_ ci
&
YES
&&
Description of RWL
\\
\hline
IR \_ descr
&
longtext
&
utf8 \_ general \_ ci
&
YES
&&
Description of IR
\\
\hline
Ref \_ indicator
&
mediumtext
&
utf8 \_ general \_ ci
&
YES
&&
Indicator reference(s)
\\
\hline
Createdby
&
varchar(45)
&
utf8 \_ general \_ ci
&
YES
&&
Record created by
\\
\hline
Updatedby
&
varchar(45)
&
utf8 \_ general \_ ci
&
YES
&&
Record updated by
\\
\hline
LastUpdate
&
date
&&
YES
&&
Last Update
\\
\hline
\end{tabular}
\par
\sphinxattableend\end{savenotes}


\section{Table: sldatum}
\label{\detokenize{database:table-sldatum}}
This table contains the sea level datums inserted in WALIS.


\begin{savenotes}\sphinxattablestart
\centering
\begin{tabular}[t]{|\X{20}{105}|\X{15}{105}|\X{25}{105}|\X{15}{105}|\X{10}{105}|\X{20}{105}|}
\hline
\sphinxstyletheadfamily 
Field
&\sphinxstyletheadfamily 
Type
&\sphinxstyletheadfamily 
Collation
&\sphinxstyletheadfamily 
Null
&\sphinxstyletheadfamily 
Key
&\sphinxstyletheadfamily 
Comment
\\
\hline
SLdatum \_ ID
&
int(11)
&&
NO
&
PRI
&
WALIS SLdatum ID
\\
\hline
SLdatumname
&
varchar(45)
&
utf8 \_ general \_ ci
&
YES
&&
Datum name
\\
\hline
SLdatumdescr
&
longtext
&
utf8 \_ general \_ ci
&
YES
&&
Datum description
\\
\hline
SLdatum \_ unc
&
longtext
&
utf8 \_ general \_ ci
&
YES
&&
Datum uncertainty
\\
\hline
Ref \_ SLdatum
&
longtext
&
utf8 \_ general \_ ci
&
YES
&&
Reference(s)
\\
\hline
Createdby
&
varchar(45)
&
utf8 \_ general \_ ci
&
YES
&&
Record created by
\\
\hline
Updatedby
&
varchar(45)
&
utf8 \_ general \_ ci
&
YES
&&
Record updated by
\\
\hline
LastUpdate
&
datetime
&&
YES
&&
Last Update
\\
\hline
\end{tabular}
\par
\sphinxattableend\end{savenotes}


\section{Table: references}
\label{\detokenize{database:table-references}}
This table contains the references inserted in WALIS.


\begin{savenotes}\sphinxattablestart
\centering
\begin{tabular}[t]{|\X{20}{105}|\X{15}{105}|\X{25}{105}|\X{15}{105}|\X{10}{105}|\X{20}{105}|}
\hline
\sphinxstyletheadfamily 
Field
&\sphinxstyletheadfamily 
Type
&\sphinxstyletheadfamily 
Collation
&\sphinxstyletheadfamily 
Null
&\sphinxstyletheadfamily 
Key
&\sphinxstyletheadfamily 
Comment
\\
\hline
Ref \_ ID
&
int(11)
&&
NO
&
PRI
&
WALIS reference ID
\\
\hline
ShortRef
&
mediumtext
&
utf8 \_ bin
&
YES
&&
Reference
\\
\hline
Fullref
&
longtext
&
utf8 \_ bin
&
YES
&&
Full reference
\\
\hline
Journal
&
mediumtext
&
utf8 \_ bin
&
YES
&&
Journal
\\
\hline
Year
&
int(4)
&&
YES
&&
Year
\\
\hline
doi
&
mediumtext
&
utf8 \_ bin
&
YES
&&
doi
\\
\hline
Link
&
varchar(100)
&
utf8 \_ general \_ ci
&
YES
&&
Link
\\
\hline
Nation
&
mediumtext
&
utf8 \_ general \_ ci
&
YES
&&
Nation
\\
\hline
Language
&
mediumtext
&
utf8 \_ bin
&
YES
&&
Language
\\
\hline
Timeframe
&
mediumtext
&
utf8 \_ bin
&
YES
&&
Timeframe
\\
\hline
Abstract
&
longtext
&
utf8 \_ bin
&
YES
&&
Abstract
\\
\hline
Createdby
&
tinytext
&
utf8 \_ general \_ ci
&
YES
&&
Record created by
\\
\hline
Updatedby
&
tinytext
&
utf8 \_ general \_ ci
&
YES
&&
Record updated by
\\
\hline
LastUpdate
&
datetime
&&
YES
&&
Last Update
\\
\hline
PaperType
&
varchar(45)
&
utf8 \_ general \_ ci
&
YES
&&
Type of paper
\\
\hline
\end{tabular}
\par
\sphinxattableend\end{savenotes}


\section{Table: Useries\_Corals}
\label{\detokenize{database:table-useries-corals}}
This table contains samples of corals, mollusks or speleothems dated with U\sphinxhyphen{}Series techniques.


\begin{savenotes}\sphinxatlongtablestart\begin{longtable}[c]{|\X{20}{105}|\X{15}{105}|\X{25}{105}|\X{15}{105}|\X{10}{105}|\X{20}{105}|}
\hline
\sphinxstyletheadfamily 
Field
&\sphinxstyletheadfamily 
Type
&\sphinxstyletheadfamily 
Collation
&\sphinxstyletheadfamily 
Null
&\sphinxstyletheadfamily 
Key
&\sphinxstyletheadfamily 
Comment
\\
\hline
\endfirsthead

\multicolumn{6}{c}%
{\makebox[0pt]{\sphinxtablecontinued{\tablename\ \thetable{} \textendash{} continued from previous page}}}\\
\hline
\sphinxstyletheadfamily 
Field
&\sphinxstyletheadfamily 
Type
&\sphinxstyletheadfamily 
Collation
&\sphinxstyletheadfamily 
Null
&\sphinxstyletheadfamily 
Key
&\sphinxstyletheadfamily 
Comment
\\
\hline
\endhead

\hline
\multicolumn{6}{r}{\makebox[0pt][r]{\sphinxtablecontinued{continues on next page}}}\\
\endfoot

\endlastfoot

ID \_ Useries
&
int(11)
&&
NO
&
PRI
&
WALIS U\sphinxhyphen{}Series ID
\\
\hline
Public
&
mediumtext
&
utf8 \_ general \_ ci
&
YES
&&
Is this datapoint public?
\\
\hline
Material \_ type
&
varchar(45)
&
utf8 \_ general \_ ci
&
YES
&&
Material type
\\
\hline
Material \_ details
&
mediumtext
&
utf8 \_ general \_ ci
&
YES
&&
Details on dated material
\\
\hline
Recalc
&
varchar(45)
&
utf8 \_ general \_ ci
&
YES
&&
Were U\sphinxhyphen{}Series data recalculated?
\\
\hline
Spectr
&
varchar(45)
&
utf8 \_ general \_ ci
&
YES
&&
Type of spectrometry
\\
\hline
RSL \_ Estimate \_ avaliable
&
varchar(45)
&
utf8 \_ general \_ ci
&
YES
&&
Are RSL estimates available for this record?
\\
\hline
Tectonics
&
varchar(45)
&
utf8 \_ general \_ ci
&
YES
&&
Are data on tectonics available?
\\
\hline
Source
&
mediumtext
&
utf8 \_ general \_ ci
&
YES
&&
Reference(s)
\\
\hline
IGSN
&
varchar(100)
&
utf8 \_ general \_ ci
&
YES
&&
IGSN
\\
\hline
SampleID
&
varchar(100)
&
utf8 \_ general \_ ci
&
YES
&&
Sample ID
\\
\hline
AnalysisID
&
varchar(100)
&
utf8 \_ general \_ ci
&
YES
&&
Analysis ID
\\
\hline
ReportedID
&
varchar(100)
&
utf8 \_ general \_ ci
&
YES
&&
Reported ID
\\
\hline
Analysis \_ Date
&
datetime
&&
YES
&&
Date of analysis
\\
\hline
Accepted
&
varchar(45)
&
utf8 \_ general \_ ci
&
YES
&&
Accepted?
\\
\hline
ReasonRej
&
longtext
&
utf8 \_ general \_ ci
&
YES
&&
Reason for rejection
\\
\hline
Accepted \_ other
&
varchar(45)
&
utf8 \_ general \_ ci
&
YES
&&
Accepted in other study?
\\
\hline
Reas \_ rej \_ other
&
longtext
&
utf8 \_ general \_ ci
&
YES
&&
Reason for rejection in other studies
\\
\hline
Other ref
&
varchar(45)
&
utf8 \_ general \_ ci
&
YES
&&
Other study ID
\\
\hline
Screening
&
longtext
&
utf8 \_ general \_ ci
&
YES
&&
Screening
\\
\hline
Location
&
mediumtext
&
utf8 \_ general \_ ci
&
YES
&&
Location
\\
\hline
Site
&
mediumtext
&
utf8 \_ general \_ ci
&
YES
&&
Site
\\
\hline
Additional \_ site \_ info
&
longtext
&
utf8 \_ general \_ ci
&
YES
&&
Additional site information
\\
\hline
SampleLat
&
varchar(45)
&
utf8 \_ general \_ ci
&
YES
&&
Latitude (decimal degrees)
\\
\hline
SampleLon
&
varchar(45)
&
utf8 \_ general \_ ci
&
YES
&&
Longitude (decimal degrees)
\\
\hline
Reported \_ Latitude
&
varchar(45)
&
utf8 \_ general \_ ci
&
YES
&&
Reported Latitude
\\
\hline
Reported \_ Longitude
&
varchar(45)
&
utf8 \_ general \_ ci
&
YES
&&
Reported Longitude
\\
\hline
Lat \_ Lon \_ est
&
mediumtext
&
utf8 \_ general \_ ci
&
YES
&&
Are Lat/Long estimated?
\\
\hline
Coord \_ comments
&
longtext
&
utf8 \_ general \_ ci
&
YES
&&
Comments on geographic coordinates
\\
\hline
Original \_ elevation \_ datum \_ used
&
mediumtext
&
utf8 \_ general \_ ci
&
YES
&&
Original elevation datum used
\\
\hline
How \_ elevation \_ derived
&
mediumtext
&
utf8 \_ general \_ ci
&
YES
&&
Elevation measurement method
\\
\hline
Elevation \_ reported
&
varchar(45)
&
utf8 \_ general \_ ci
&
YES
&&
Reported elevation (m)
\\
\hline
Reported \_ elevation \_ uncertainty
&
varchar(45)
&
utf8 \_ general \_ ci
&
YES
&&
Reported elevation uncertainty (m)
\\
\hline
Elevation \_ above \_ MSL
&
varchar(45)
&
utf8 \_ general \_ ci
&
YES
&&
Elevation above MSL (m)
\\
\hline
Elevation \_ uncertainty \_ used
&
varchar(45)
&
utf8 \_ general \_ ci
&
YES
&&
Elevation uncertainty used (m)
\\
\hline
Interpr \_ elev
&
mediumtext
&
utf8 \_ general \_ ci
&
YES
&&
Interpreted elevation relative to mllw/mlws (m)
\\
\hline
Elevation \_ comments
&
longtext
&
utf8 \_ general \_ ci
&
YES
&&
Elevation comments
\\
\hline
Tect \_ cat
&
mediumtext
&
utf8 \_ general \_ ci
&
YES
&&
Tectonic category
\\
\hline
Publ \_ uplift
&
varchar(45)
&
utf8 \_ general \_ ci
&
YES
&&
Published uplift rate (m/ky)
\\
\hline
Publ \_ uplift \_ unc
&
varchar(45)
&
utf8 \_ general \_ ci
&
YES
&&
Published uncertainty in uplift rate (m/ky)
\\
\hline
Interpr \_ uplift
&
varchar(45)
&
utf8 \_ general \_ ci
&
YES
&&
Interpreted uplift rate (m/ky)
\\
\hline
Interpr \_ uplift \_ unc
&
varchar(45)
&
utf8 \_ general \_ ci
&
YES
&&
Interpreted uplift rate uncertainty (m/ky)
\\
\hline
Comments \_ uplift
&
longtext
&
utf8 \_ general \_ ci
&
YES
&&
Comments (uplift), including sources of uplift rates
\\
\hline
Speleo \_ insitu
&
varchar(45)
&
utf8 \_ general \_ ci
&
YES
&&
Reported in situ?
\\
\hline
Speleo \_ deposit \_ type
&
varchar(45)
&
utf8 \_ general \_ ci
&
YES
&&
Type of deposit
\\
\hline
Speleo \_ distance \_ base
&
varchar(45)
&
utf8 \_ general \_ ci
&
YES
&&
Distance from base of deposit (m)
\\
\hline
Speleo \_ Sampled \_ material
&
varchar(45)
&
utf8 \_ general \_ ci
&
YES
&&
Sampled material
\\
\hline
Speleo \_ add \_ context
&
longtext
&
utf8 \_ general \_ ci
&
YES
&&
Additional sample context
\\
\hline
Speleo \_ RSL
&
varchar(45)
&
utf8 \_ general \_ ci
&
YES
&&
paleo RSL (m)
\\
\hline
Speleo \_ RSL \_ uncertainty
&
varchar(45)
&
utf8 \_ general \_ ci
&
YES
&&
paleo RSL uncertainty (m)
\\
\hline
Terrace \_ ID
&
mediumtext
&
utf8 \_ general \_ ci
&
YES
&&
Terrace ID
\\
\hline
Facies \_ descr
&
longtext
&
utf8 \_ general \_ ci
&
YES
&&
Facies description
\\
\hline
In \_ situ
&
tinytext
&
utf8 \_ general \_ ci
&
YES
&&
Reported as in situ?
\\
\hline
Growth \_ pos
&
tinytext
&
utf8 \_ general \_ ci
&
YES
&&
Reported as in growth position?
\\
\hline
Coral \_ taxa \_ information
&
longtext
&
utf8 \_ general \_ ci
&
YES
&&
Taxa information (as reported)
\\
\hline
Family
&
mediumtext
&
utf8 \_ general \_ ci
&
YES
&&
Family
\\
\hline
Genus
&
mediumtext
&
utf8 \_ general \_ ci
&
YES
&&
Genus
\\
\hline
Species
&
mediumtext
&
utf8 \_ general \_ ci
&
YES
&&
Species
\\
\hline
Original \_ paleodepth \_ interpretation
&
longtext
&
utf8 \_ general \_ ci
&
YES
&&
Original palaeodepth interpretation
\\
\hline
Comments \_ Ecology
&
longtext
&
utf8 \_ general \_ ci
&
YES
&&
Ecological metadata
\\
\hline
PWD \_ estimate
&
varchar(45)
&
utf8 \_ general \_ ci
&
YES
&&
Paleo water depth estimate (m)
\\
\hline
PWD \_ upper
&
varchar(45)
&
utf8 \_ general \_ ci
&
YES
&&
Upper limit of living range (m)
\\
\hline
PWD \_ lower
&
varchar(45)
&
utf8 \_ general \_ ci
&
YES
&&
Lower limit of living range (m)
\\
\hline
PWD \_ comments
&
longtext
&
utf8 \_ general \_ ci
&
YES
&&
Paleo water depth comments
\\
\hline
Pa \_ Th \_ Age
&
tinytext
&
utf8 \_ general \_ ci
&
YES
&&
Pa/Th age?
\\
\hline
14C \_ Age
&
tinytext
&
utf8 \_ general \_ ci
&
YES
&&
14C age?
\\
\hline
Instrument
&
varchar(45)
&
utf8 \_ general \_ ci
&
YES
&&
Instrument
\\
\hline
Decay \_ Const
&
varchar(45)
&
utf8 \_ general \_ ci
&
YES
&&
Decay constants
\\
\hline
Comments \_ on \_ decay \_ constants
&
longtext
&
utf8 \_ general \_ ci
&
YES
&&
Comments on decay constants
\\
\hline
Calib \_ Method \_ 230Th \_ 238U
&
varchar(45)
&
utf8 \_ general \_ ci
&
YES
&&
Calibration method for 230Th/238U ratio
\\
\hline
Calib \_ Method \_ 234U \_ 238U
&
varchar(45)
&
utf8 \_ general \_ ci
&
YES
&&
Calibration method for 234U/238U ratio
\\
\hline
Comments \_ on \_ spike \_ calibration
&
longtext
&
utf8 \_ general \_ ci
&
YES
&&
Comments on spike calibration
\\
\hline
Speleo \_ mineralogy \_ detected
&
varchar(45)
&
utf8 \_ general \_ ci
&
YES
&&
How was speleothem mineralogy determined?
\\
\hline
Speleo \_ mineralogy
&
varchar(45)
&
utf8 \_ general \_ ci
&
YES
&&
Speleothem mineralogy
\\
\hline
Other \_ screening \_ techniques
&
longtext
&
utf8 \_ general \_ ci
&
YES
&&
Other screening techniques applied
\\
\hline
Publ \_ calcite
&
varchar(45)
&
utf8 \_ general \_ ci
&
YES
&&
Published \% calcite
\\
\hline
Interpr \_ calcite
&
varchar(45)
&
utf8 \_ general \_ ci
&
YES
&&
Interpreted \% calcite
\\
\hline
230 \_ 232 \_ Th \_ ACT \_ backcalc
&
varchar(45)
&
utf8 \_ general \_ ci
&
YES
&&
{[}230Th/232Th{]}ACT backcalculated?
\\
\hline
232Th \_ 238U \_ ACT \_ backcalc
&
varchar(45)
&
utf8 \_ general \_ ci
&
YES
&&
{[}232Th/238U{]}ACT backcalculated?
\\
\hline
230Th \_ 238U \_ ACT \_ backcalc
&
varchar(45)
&
utf8 \_ general \_ ci
&
YES
&&
{[}230Th/238U{]}ACT backcalculated?
\\
\hline
234Th \_ 238U \_ ACT \_ backcalc
&
varchar(45)
&
utf8 \_ general \_ ci
&
YES
&&
{[}234Th/238U{]}ACT backcalculated?
\\
\hline
232Th
&
varchar(45)
&
utf8 \_ general \_ ci
&
YES
&&
{[}232Th{]} (ppb)
\\
\hline
232Th \_ unc
&
varchar(45)
&
utf8 \_ general \_ ci
&
YES
&&
{[}232Th{]} (ppb) uncertainty (\(\pm\)2sigma)
\\
\hline
238U
&
varchar(45)
&
utf8 \_ general \_ ci
&
YES
&&
{[}238U{]} (ppm)
\\
\hline
238U \_ unc
&
varchar(45)
&
utf8 \_ general \_ ci
&
YES
&&
{[}238U{]} (ppm) uncertainty (\(\pm\)2sigma)
\\
\hline
Atom \_ ratio \_ 232Th \_ 238U
&
varchar(45)
&
utf8 \_ general \_ ci
&
YES
&&
Atomic ratio (232Th/238U)*10\textasciicircum{}5
\\
\hline
230 \_ 232Th \_ init
&
varchar(45)
&
utf8 \_ general \_ ci
&
YES
&&
Initial 230Th/232Th
\\
\hline
230Th \_ 232Th \_ ACT
&
varchar(45)
&
utf8 \_ general \_ ci
&
YES
&&
{[}230Th/232Th{]}ACT
\\
\hline
230Th \_ 232Th \_ ACT \_ unc
&
varchar(45)
&
utf8 \_ general \_ ci
&
YES
&&
{[}230Th/232Th{]}ACT uncertainty (\(\pm\)2sigma)
\\
\hline
232Th \_ 238U \_ ACT
&
varchar(45)
&
utf8 \_ general \_ ci
&
YES
&&
{[}232Th/238U{]}ACT
\\
\hline
232Th \_ 238U \_ ACT \_ unc
&
varchar(45)
&
utf8 \_ general \_ ci
&
YES
&&
{[}232Th/238U{]}ACT uncertainty (\(\pm\)2sigma)
\\
\hline
230Th \_ 234U \_ ACT
&
varchar(45)
&
utf8 \_ general \_ ci
&
YES
&&
{[}230Th/234U{]}ACT
\\
\hline
230Th \_ 234U \_ ACT \_ unc
&
varchar(45)
&
utf8 \_ general \_ ci
&
YES
&&
{[}230Th/234U{]}ACT uncertainty (\(\pm\)2sigma)
\\
\hline
230Th \_ 238U \_ ACT
&
varchar(45)
&
utf8 \_ general \_ ci
&
YES
&&
{[}230Th/238U{]}ACT
\\
\hline
230Th \_ 238U \_ ACT \_ unc
&
varchar(45)
&
utf8 \_ general \_ ci
&
YES
&&
{[}230Th/238U{]}ACT uncertainty (\(\pm\)2sigma)
\\
\hline
234U \_ 238 \_ ACT
&
varchar(45)
&
utf8 \_ general \_ ci
&
YES
&&
{[}234U/238U{]}ACT
\\
\hline
234U \_ 238U \_ ACT \_ unc
&
varchar(45)
&
utf8 \_ general \_ ci
&
YES
&&
{[}234U/238U{]}ACT uncertainty (\(\pm\)2sigma)
\\
\hline
Reported \_ age
&
varchar(45)
&
utf8 \_ general \_ ci
&
YES
&&
Reported age (ka)
\\
\hline
Reported \_ age \_ unc
&
varchar(45)
&
utf8 \_ general \_ ci
&
YES
&&
Reported ageuncertainty (ka, \(\pm\)2sigma)
\\
\hline
Speleo \_ corr \_ rep \_ age
&
varchar(45)
&
utf8 \_ general \_ ci
&
YES
&&
Corrected reported age (ka)
\\
\hline
Speleo \_ corr \_ rep \_ age \_ unc
&
varchar(45)
&
utf8 \_ general \_ ci
&
YES
&&
Corrected reported age uncert. (ka, \(\pm\)2sigma)
\\
\hline
Reported \_ d234U \_ init
&
varchar(45)
&
utf8 \_ general \_ ci
&
YES
&&
Reported delta 234U initial (‰)
\\
\hline
Reported \_ d234U \_ unc
&
varchar(45)
&
utf8 \_ general \_ ci
&
YES
&&
Reported delta 234U (‰) uncertainty (\(\pm\)2sigma)
\\
\hline
234U \_ 238U \_ init
&
varchar(45)
&
utf8 \_ general \_ ci
&
YES
&&
Initial 234U/238U
\\
\hline
234U \_ 238U \_ init \_ unc
&
varchar(45)
&
utf8 \_ general \_ ci
&
YES
&&
Initial 234U/238U uncertainty (\(\pm\)2sigma)
\\
\hline
234U \_ 238U \_ meas
&
varchar(45)
&
utf8 \_ general \_ ci
&
YES
&&
Measured 234U/238U
\\
\hline
234U \_ 238U \_ meas \_ unc
&
varchar(45)
&
utf8 \_ general \_ ci
&
YES
&&
Measured 234U/238U uncertainty (\(\pm\)2sigma)
\\
\hline
Meas \_ delta234U
&
varchar(45)
&
utf8 \_ general \_ ci
&
YES
&&
Measured delta234U
\\
\hline
Meas \_ delta234U \_ unc
&
varchar(45)
&
utf8 \_ general \_ ci
&
YES
&&
Measured delta234U uncertainty (\(\pm\)2sigma)
\\
\hline
Comments \_ Age
&
longtext
&
utf8 \_ general \_ ci
&
YES
&&
Comments on age determination
\\
\hline
Ref \_ 230Th \_ 238U
&
varchar(45)
&
utf8 \_ general \_ ci
&
YES
&&
Reference material name for 230Th/238U
\\
\hline
Ref \_ 234U \_ 238U
&
varchar(45)
&
utf8 \_ general \_ ci
&
YES
&&
Reference material name for 234U/238U
\\
\hline
Corr \_ 230Th \_ 238U
&
varchar(45)
&
utf8 \_ general \_ ci
&
YES
&&
Correction factor for 230Th/238U
\\
\hline
Corr \_ 230Th \_ 238U \_ unc
&
varchar(45)
&
utf8 \_ general \_ ci
&
YES
&&
Correction factor for 230Th/238U uncertainty (\(\pm\)2sigma)
\\
\hline
Corr \_ 234U \_ 238U
&
varchar(45)
&
utf8 \_ general \_ ci
&
YES
&&
Correction factor for 234U/238U
\\
\hline
Corr \_ 234U \_ 238U \_ unc
&
varchar(45)
&
utf8 \_ general \_ ci
&
YES
&&
Correction factor for 234U/238U uncertainty (\(\pm\)2sigma)
\\
\hline
Comments
&
longtext
&
utf8 \_ general \_ ci
&
YES
&&
Comments
\\
\hline
Rec \_ 230Th \_ 238U \_ ACT
&
varchar(45)
&
utf8 \_ general \_ ci
&
YES
&&
Recalculated {[}230Th/238U{]}ACT
\\
\hline
Rec \_ 230Th \_ 238U \_ ACT \_ unc
&
varchar(45)
&
utf8 \_ general \_ ci
&
YES
&&
Recalculated {[}230Th/238U{]}ACT uncertainty (\(\pm\)2sigma)
\\
\hline
Rec \_ 234U \_ 238 \_ ACT
&
varchar(45)
&
utf8 \_ general \_ ci
&
YES
&&
Recalculated {[}234U/238U{]}ACT
\\
\hline
Rec \_ 234U \_ 238U \_ ACT \_ unc
&
varchar(45)
&
utf8 \_ general \_ ci
&
YES
&&
Recalculated {[}234U/238U{]}ACT uncertainty (\(\pm\)2sigma)
\\
\hline
Rec \_ age
&
varchar(45)
&
utf8 \_ general \_ ci
&
YES
&&
Recalculated Conventional Age (ka)
\\
\hline
Rec \_ age \_ unc
&
varchar(45)
&
utf8 \_ general \_ ci
&
YES
&&
Recalculated Conventional Age uncert. (\(\pm\)2sigma)
\\
\hline
Rec \_ d234U \_ init
&
varchar(45)
&
utf8 \_ general \_ ci
&
YES
&&
Recalculated delta 234Ui (‰)
\\
\hline
Rec \_ d234U \_ unc
&
varchar(45)
&
utf8 \_ general \_ ci
&
YES
&&
Recalculated delta 234Ui uncertainty (\(\pm\)2sigma)
\\
\hline
Rec \_ Conv \_ age \_ no \_ decay
&
varchar(45)
&
utf8 \_ general \_ ci
&
YES
&&
Recalculated Conventional Age uncert. w/ decay constant uncertainties (\(\pm\)2sigma)
\\
\hline
Rec \_ d234Init \_ no \_ decay
&
varchar(45)
&
utf8 \_ general \_ ci
&
YES
&&
Recalculated delta 234Ui uncert. (\(\pm\)2sigma) w/ decay constant uncertainties
\\
\hline
Comments age and d234Ui
&
longtext
&
utf8 \_ general \_ ci
&
YES
&&
Comments (age and delta234i)
\\
\hline
MISchoice \_ limit
&
varchar(45)
&
utf8 \_ general \_ ci
&
YES
&&
Age is Older/Equal/Younger than
\\
\hline
MISchoice
&
varchar(45)
&
utf8 \_ general \_ ci
&
YES
&&
Marine Isotopic Stage
\\
\hline
Age determination
&
longtext
&
utf8 \_ general \_ ci
&
YES
&&
Age determination
\\
\hline
Createdby
&
varchar(45)
&
utf8 \_ general \_ ci
&
YES
&&
Record created by
\\
\hline
Updatedby
&
varchar(45)
&
utf8 \_ general \_ ci
&
YES
&&
Record updated by
\\
\hline
LastUpdate
&
datetime
&&
YES
&&
Last Update
\\
\hline
coord
&
point
&&
YES
&&
MySQL coordinates
\\
\hline
\end{longtable}\sphinxatlongtableend\end{savenotes}


\section{Table: aar}
\label{\detokenize{database:table-aar}}
This table contains samples dated with Amino Acid Racemization.


\begin{savenotes}\sphinxatlongtablestart\begin{longtable}[c]{|\X{20}{105}|\X{15}{105}|\X{25}{105}|\X{15}{105}|\X{10}{105}|\X{20}{105}|}
\hline
\sphinxstyletheadfamily 
Field
&\sphinxstyletheadfamily 
Type
&\sphinxstyletheadfamily 
Collation
&\sphinxstyletheadfamily 
Null
&\sphinxstyletheadfamily 
Key
&\sphinxstyletheadfamily 
Comment
\\
\hline
\endfirsthead

\multicolumn{6}{c}%
{\makebox[0pt]{\sphinxtablecontinued{\tablename\ \thetable{} \textendash{} continued from previous page}}}\\
\hline
\sphinxstyletheadfamily 
Field
&\sphinxstyletheadfamily 
Type
&\sphinxstyletheadfamily 
Collation
&\sphinxstyletheadfamily 
Null
&\sphinxstyletheadfamily 
Key
&\sphinxstyletheadfamily 
Comment
\\
\hline
\endhead

\hline
\multicolumn{6}{r}{\makebox[0pt][r]{\sphinxtablecontinued{continues on next page}}}\\
\endfoot

\endlastfoot

ID AAR
&
int(11)
&&
NO
&
PRI
&
WALIS AAR ID
\\
\hline
Public
&
varchar(45)
&
utf8 \_ general \_ ci
&
YES
&&
Is this datapoint public?
\\
\hline
AARRef
&
mediumtext
&
utf8 \_ general \_ ci
&
YES
&&
Reference(s)
\\
\hline
IGSN
&
varchar(45)
&
utf8 \_ general \_ ci
&
YES
&&
IGSN
\\
\hline
SampleID
&
varchar(100)
&
utf8 \_ general \_ ci
&
YES
&&
Sample ID
\\
\hline
AnalysisID
&
varchar(100)
&
utf8 \_ general \_ ci
&
YES
&&
Analysis ID
\\
\hline
ReportedID
&
varchar(100)
&
utf8 \_ general \_ ci
&
YES
&&
Reported ID
\\
\hline
Analysis \_ Date
&
datetime
&&
YES
&&
Date of analysis
\\
\hline
Laboratory name
&
mediumtext
&
utf8 \_ general \_ ci
&
YES
&&
Laboratory name
\\
\hline
SampleType
&
longtext
&
utf8 \_ general \_ ci
&
YES
&&
Sample Type
\\
\hline
Taxonomy
&
longtext
&
utf8 \_ general \_ ci
&
YES
&&
Taxonomy
\\
\hline
SamplePos
&
longtext
&
utf8 \_ general \_ ci
&
YES
&&
Sample position
\\
\hline
WRock
&
int(11)
&&
YES
&&
Whole\sphinxhyphen{}Rock?
\\
\hline
WRsize
&
varchar(45)
&
utf8 \_ general \_ ci
&
YES
&&
Grain Size (micrometersm)
\\
\hline
WRcarb
&
varchar(45)
&
utf8 \_ general \_ ci
&
YES
&&
Carbonate (\%)
\\
\hline
WRfract
&
varchar(45)
&
utf8 \_ general \_ ci
&
YES
&&
Fraction (micrometersm)
\\
\hline
SampleQualNotes
&
longtext
&
utf8 \_ general \_ ci
&
YES
&&
Notes and comments on sample quality
\\
\hline
CollectionContext
&
longtext
&
utf8 \_ general \_ ci
&
YES
&&
Collection Context
\\
\hline
DoB
&
varchar(45)
&
utf8 \_ general \_ ci
&
YES
&&
Depth of burial (m)
\\
\hline
Accepted
&
varchar(45)
&
utf8 \_ general \_ ci
&
YES
&&
Accepted?
\\
\hline
ReasRej
&
longtext
&
utf8 \_ general \_ ci
&
YES
&&
Reason for rejection
\\
\hline
Reported \_ Latitude
&
varchar(45)
&
utf8 \_ general \_ ci
&
YES
&&
Reported Latitude
\\
\hline
Reported \_ Longitude
&
varchar(45)
&
utf8 \_ general \_ ci
&
YES
&&
Reported Longitude
\\
\hline
SampleLat
&
varchar(45)
&
utf8 \_ general \_ ci
&
YES
&&
Latitude (decimal degrees)
\\
\hline
SampleLon
&
varchar(45)
&
utf8 \_ general \_ ci
&
YES
&&
Longitude (decimal degrees)
\\
\hline
Lat \_ Lon \_ Est
&
varchar(45)
&
utf8 \_ general \_ ci
&
YES
&&
Are Lat/Long estimated?
\\
\hline
Original \_ datum
&
varchar(45)
&
utf8 \_ general \_ ci
&
YES
&&
Original elevation datum used
\\
\hline
Elev \_ meas \_ method
&
varchar(45)
&
utf8 \_ general \_ ci
&
YES
&&
Elevation measurement method
\\
\hline
Reported \_ Elev
&
varchar(45)
&
utf8 \_ general \_ ci
&
YES
&&
Reported elevation (m)
\\
\hline
Reported \_ Elev \_ Unc
&
varchar(45)
&
utf8 \_ general \_ ci
&
YES
&&
Reported elevation uncertainty (m)
\\
\hline
Elev \_ MSL
&
varchar(45)
&
utf8 \_ general \_ ci
&
YES
&&
Elevation above MSL (m)
\\
\hline
Elev \_ MSL \_ unc
&
varchar(45)
&
utf8 \_ general \_ ci
&
YES
&&
Elevation uncertainty used (m)
\\
\hline
Elev \_ comments
&
longtext
&
utf8 \_ general \_ ci
&
YES
&&
Elevation comments
\\
\hline
MISchoice \_ limit
&
varchar(45)
&
utf8 \_ general \_ ci
&
YES
&&
Age is Older/Equal/Younger than
\\
\hline
MISchoice
&
varchar(45)
&
utf8 \_ general \_ ci
&
YES
&&
Marine Isotopic Stage
\\
\hline
MIS
&
longtext
&
utf8 \_ general \_ ci
&
YES
&&
Comments/details on MIS designation
\\
\hline
ReportedAge
&
varchar(45)
&
utf8 \_ general \_ ci
&
YES
&&
Reported age (ka)
\\
\hline
ReportedAgeUnc
&
varchar(45)
&
utf8 \_ general \_ ci
&
YES
&&
Reported age uncertainty (\(\pm\)2sigma) (ka)
\\
\hline
CMAT
&
varchar(45)
&
utf8 \_ general \_ ci
&
YES
&&
Current Annual Mean Temperature (°C)
\\
\hline
PrepProc
&
longtext
&
utf8 \_ general \_ ci
&
YES
&&
Preparation procedure
\\
\hline
ChromMethod
&
longtext
&
utf8 \_ general \_ ci
&
YES
&&
Method of chromatography
\\
\hline
StatPhaseCol
&
longtext
&
utf8 \_ general \_ ci
&
YES
&&
Stationary phase of the column
\\
\hline
ReplNsamples
&
varchar(45)
&
utf8 \_ general \_ ci
&
YES
&&
Number of replicates (samples)
\\
\hline
ReplNinj
&
varchar(45)
&
utf8 \_ general \_ ci
&
YES
&&
Number of sample replicates (injections)
\\
\hline
RatioType
&
mediumtext
&
utf8 \_ general \_ ci
&
YES
&&
Ratio type
\\
\hline
PyrolysisAvail
&
mediumtext
&
utf8 \_ general \_ ci
&
YES
&&
Pyrolysis experiment available
\\
\hline
LevUnc
&
varchar(45)
&
utf8 \_ general \_ ci
&
YES
&&
Level of uncertainty
\\
\hline
Aspartic
&
varchar(45)
&
utf8 \_ general \_ ci
&
YES
&&
Aspartic
\\
\hline
Aspartic \_ unc
&
varchar(45)
&
utf8 \_ general \_ ci
&
YES
&&
Aspartic uncertainty
\\
\hline
Glutamic
&
varchar(45)
&
utf8 \_ general \_ ci
&
YES
&&
Glutamic
\\
\hline
Glutamic \_ unc
&
varchar(45)
&
utf8 \_ general \_ ci
&
YES
&&
Glutamic uncertainty
\\
\hline
Serine
&
varchar(45)
&
utf8 \_ general \_ ci
&
YES
&&
Serine
\\
\hline
Serine \_ unc
&
varchar(45)
&
utf8 \_ general \_ ci
&
YES
&&
Serine uncertainty
\\
\hline
Alanine
&
varchar(45)
&
utf8 \_ general \_ ci
&
YES
&&
Alanine
\\
\hline
Alanine \_ unc
&
varchar(45)
&
utf8 \_ general \_ ci
&
YES
&&
Alanine uncertainty
\\
\hline
Valine
&
varchar(45)
&
utf8 \_ general \_ ci
&
YES
&&
Valine
\\
\hline
Valine \_ unc
&
varchar(45)
&
utf8 \_ general \_ ci
&
YES
&&
Valine uncertainty
\\
\hline
A \_ I
&
varchar(45)
&
utf8 \_ general \_ ci
&
YES
&&
{\color{red}\bfseries{}*}A/I
\\
\hline
AI \_ unc
&
varchar(45)
&
utf8 \_ general \_ ci
&
YES
&&
{\color{red}\bfseries{}*}A/I uncertainty
\\
\hline
Leucine
&
varchar(45)
&
utf8 \_ general \_ ci
&
YES
&&
Leucine
\\
\hline
Leucine \_ unc
&
varchar(45)
&
utf8 \_ general \_ ci
&
YES
&&
Leucine uncertainty
\\
\hline
Notes
&
longtext
&
utf8 \_ general \_ ci
&
YES
&&
Notes/Comments
\\
\hline
ShowIndep
&
varchar(45)
&
utf8 \_ general \_ ci
&
YES
&&
Independent age constraints available?
\\
\hline
Showblock
&
varchar(45)
&
utf8 \_ general \_ ci
&
YES
&&
Calibration data available?
\\
\hline
freeAAR
&
varchar(45)
&
utf8 \_ general \_ ci
&
YES
&&
Are free AAR ratios available?
\\
\hline
LevUnc \_ FAA
&
varchar(45)
&
utf8 \_ general \_ ci
&
YES
&&
Level of uncertainty free AAR
\\
\hline
Aspartic \_ FAA
&
varchar(45)
&
utf8 \_ general \_ ci
&
YES
&&
Aspartic (free AAR)
\\
\hline
Aspartic \_ unc \_ FAA
&
varchar(45)
&
utf8 \_ general \_ ci
&
YES
&&
Aspartic uncertainty (free AAR)
\\
\hline
Glutamic \_ FAA
&
varchar(45)
&
utf8 \_ general \_ ci
&
YES
&&
Glutamic (free AAR)
\\
\hline
Glutamic \_ unc \_ FAA
&
varchar(45)
&
utf8 \_ general \_ ci
&
YES
&&
Glutamic uncertainty (free AAR)
\\
\hline
Serine \_ FAA
&
varchar(45)
&
utf8 \_ general \_ ci
&
YES
&&
Serine (free AAR)
\\
\hline
Serine \_ unc \_ FAA
&
varchar(45)
&
utf8 \_ general \_ ci
&
YES
&&
Serine uncertainty (free AAR)
\\
\hline
Alanine \_ FAA
&
varchar(45)
&
utf8 \_ general \_ ci
&
YES
&&
Alanine (free AAR)
\\
\hline
Alanine \_ unc \_ FAA
&
varchar(45)
&
utf8 \_ general \_ ci
&
YES
&&
Alanine uncertainty (free AAR)
\\
\hline
Valine \_ FAA
&
varchar(45)
&
utf8 \_ general \_ ci
&
YES
&&
Valine (free AAR)
\\
\hline
Valine \_ unc \_ FAA
&
varchar(45)
&
utf8 \_ general \_ ci
&
YES
&&
Valine uncertainty (free AAR)
\\
\hline
A \_ I \_ FAA
&
varchar(45)
&
utf8 \_ general \_ ci
&
YES
&&
{\color{red}\bfseries{}*}A/I (free AAR)
\\
\hline
AI \_ unc \_ FAA
&
varchar(45)
&
utf8 \_ general \_ ci
&
YES
&&
{\color{red}\bfseries{}*}A/I uncertainty (free AAR)
\\
\hline
Leucine \_ FAA
&
varchar(45)
&
utf8 \_ general \_ ci
&
YES
&&
Leucine (free AAR)
\\
\hline
Leucine \_ unc \_ FAA
&
varchar(45)
&
utf8 \_ general \_ ci
&
YES
&&
Leucine uncertainty (free AAR)
\\
\hline
Notes \_ FAA
&
longtext
&
utf8 \_ general \_ ci
&
YES
&&
Notes/Comments free AAR
\\
\hline
CalibMethod
&
longtext
&
utf8 \_ general \_ ci
&
YES
&&
Calibration method
\\
\hline
AARequation
&
mediumtext
&
utf8 \_ general \_ ci
&
YES
&&
Amino Acid age equation
\\
\hline
CalibLab
&
mediumtext
&
utf8 \_ general \_ ci
&
YES
&&
Calibration laboratory
\\
\hline
CalibLabID
&
mediumtext
&
utf8 \_ general \_ ci
&
YES
&&
Calibration laboratory ID
\\
\hline
CalibRef
&
mediumtext
&
utf8 \_ general \_ ci
&
YES
&&
Calibration reference
\\
\hline
CalibDLvalue
&
varchar(45)
&
utf8 \_ general \_ ci
&
YES
&&
Calibration D/L value
\\
\hline
CalibDLvalue \_ unc
&
varchar(45)
&
utf8 \_ general \_ ci
&
YES
&&
Calibration D/L value uncertainty (\(\pm\)1sigma)
\\
\hline
Calib \_ Age
&
varchar(45)
&
utf8 \_ general \_ ci
&
YES
&&
Calibration age (ka)
\\
\hline
Calib \_ Age \_ unc
&
varchar(45)
&
utf8 \_ general \_ ci
&
YES
&&
Calibration age (ka) uncertainty (\(\pm\)1sigma)
\\
\hline
Indep \_ age \_ choice
&
varchar(45)
&
utf8 \_ general \_ ci
&
YES
&&
Independent age choice
\\
\hline
IndepAgeMethod
&
longtext
&
utf8 \_ general \_ ci
&
YES
&&
Independent age method
\\
\hline
IndepAgeLab
&
mediumtext
&
utf8 \_ general \_ ci
&
YES
&&
Independent age laboratory
\\
\hline
IndepAgeLabID
&
mediumtext
&
utf8 \_ general \_ ci
&
YES
&&
Independent age laboratory ID
\\
\hline
IndepAgeResult
&
longtext
&
utf8 \_ general \_ ci
&
YES
&&
Independent age result
\\
\hline
IndepAgeRef
&
mediumtext
&
utf8 \_ general \_ ci
&
YES
&&
Independent age reference
\\
\hline
Useries constraint
&
varchar(45)
&
utf8 \_ general \_ ci
&
YES
&&
U\sphinxhyphen{}Series age constraint
\\
\hline
ESR constraint
&
varchar(45)
&
utf8 \_ general \_ ci
&
YES
&&
ESR age constraint
\\
\hline
Luminescene constraint
&
varchar(45)
&
utf8 \_ general \_ ci
&
YES
&&
Luminescence age constraint
\\
\hline
Strat constraint
&
varchar(45)
&
utf8 \_ general \_ ci
&
YES
&&
Stratigraphic age constraint
\\
\hline
Createdby
&
varchar(45)
&
utf8 \_ general \_ ci
&
YES
&&
Record created by
\\
\hline
Updatedby
&
varchar(45)
&
utf8 \_ general \_ ci
&
YES
&&
Record updated by
\\
\hline
LastUpdate
&
datetime
&&
YES
&&
Last Update
\\
\hline
coord
&
point
&&
YES
&&
mySQL coordinates
\\
\hline
\end{longtable}\sphinxatlongtableend\end{savenotes}


\section{Table: esr}
\label{\detokenize{database:table-esr}}
This table contains samples dated with Electron Spin Resonance.


\begin{savenotes}\sphinxatlongtablestart\begin{longtable}[c]{|\X{20}{105}|\X{15}{105}|\X{25}{105}|\X{15}{105}|\X{10}{105}|\X{20}{105}|}
\hline
\sphinxstyletheadfamily 
Field
&\sphinxstyletheadfamily 
Type
&\sphinxstyletheadfamily 
Collation
&\sphinxstyletheadfamily 
Null
&\sphinxstyletheadfamily 
Key
&\sphinxstyletheadfamily 
Comment
\\
\hline
\endfirsthead

\multicolumn{6}{c}%
{\makebox[0pt]{\sphinxtablecontinued{\tablename\ \thetable{} \textendash{} continued from previous page}}}\\
\hline
\sphinxstyletheadfamily 
Field
&\sphinxstyletheadfamily 
Type
&\sphinxstyletheadfamily 
Collation
&\sphinxstyletheadfamily 
Null
&\sphinxstyletheadfamily 
Key
&\sphinxstyletheadfamily 
Comment
\\
\hline
\endhead

\hline
\multicolumn{6}{r}{\makebox[0pt][r]{\sphinxtablecontinued{continues on next page}}}\\
\endfoot

\endlastfoot

ESR \_ ID
&
int(11)
&&
NO
&
PRI
&
WALIS ESR ID
\\
\hline
Public
&
varchar(45)
&
utf8 \_ general \_ ci
&
YES
&&
Is this datapoint public?
\\
\hline
ESR \_ Method
&
tinytext
&
utf8 \_ general \_ ci
&
YES
&&
Method
\\
\hline
IGSN
&
varchar(45)
&
utf8 \_ general \_ ci
&
YES
&&
IGSN
\\
\hline
Refs
&
longtext
&
utf8 \_ general \_ ci
&
YES
&&
Reference(s)
\\
\hline
Sample \_ ID
&
varchar(45)
&
utf8 \_ general \_ ci
&
YES
&&
Sample ID
\\
\hline
AnalysisID
&
varchar(45)
&
utf8 \_ general \_ ci
&
YES
&&
Analysis ID
\\
\hline
ReportedID
&
varchar(45)
&
utf8 \_ general \_ ci
&
YES
&&
Reported ID
\\
\hline
Lab \_ name
&
mediumtext
&
utf8 \_ general \_ ci
&
YES
&&
Laboratory name
\\
\hline
AnalysisDate
&
datetime
&&
YES
&&
Date of analysis
\\
\hline
Accepted
&
varchar(45)
&
utf8 \_ general \_ ci
&
YES
&&
Accepted?
\\
\hline
Reas \_ rej
&
longtext
&
utf8 \_ general \_ ci
&
YES
&&
Reason for rejection
\\
\hline
Reported \_ Latitude
&
varchar(45)
&
utf8 \_ general \_ ci
&
YES
&&
Reported Latitude
\\
\hline
Reported \_ Longitude
&
varchar(45)
&
utf8 \_ general \_ ci
&
YES
&&
Reported Longitude
\\
\hline
SampleLat
&
varchar(45)
&
utf8 \_ general \_ ci
&
YES
&&
Latitude (decimal degrees)
\\
\hline
SampleLon
&
varchar(45)
&
utf8 \_ general \_ ci
&
YES
&&
Longitude (decimal degrees)
\\
\hline
Lat \_ Lon \_ Est
&
varchar(45)
&
utf8 \_ general \_ ci
&
YES
&&
Are Lat/Long estimated?
\\
\hline
Original \_ datum
&
varchar(45)
&
utf8 \_ general \_ ci
&
YES
&&
Original elevation datum used
\\
\hline
Elev \_ meas \_ method
&
varchar(45)
&
utf8 \_ general \_ ci
&
YES
&&
Elevation measurement method
\\
\hline
Reported \_ Elev
&
varchar(45)
&
utf8 \_ general \_ ci
&
YES
&&
Reported elevation (m)
\\
\hline
Reported \_ Elev \_ Unc
&
varchar(45)
&
utf8 \_ general \_ ci
&
YES
&&
Reported elevation uncertainty (m)
\\
\hline
Elev \_ MSL
&
varchar(45)
&
utf8 \_ general \_ ci
&
YES
&&
Elevation above MSL (m)
\\
\hline
Elev \_ MSL \_ unc
&
varchar(45)
&
utf8 \_ general \_ ci
&
YES
&&
Elevation uncertainty used (m)
\\
\hline
Elev \_ comments
&
longtext
&
utf8 \_ general \_ ci
&
YES
&&
Elevation comments
\\
\hline
MISchoice \_ limit
&
varchar(45)
&
utf8 \_ general \_ ci
&
YES
&&
Age is Older/Younger/Equal to
\\
\hline
MISchoice
&
varchar(45)
&
utf8 \_ general \_ ci
&
YES
&&
Marine Isotopic Stage
\\
\hline
Reported \_ MIS
&
longtext
&
utf8 \_ general \_ ci
&
YES
&&
Comments/details on MIS designation or age
\\
\hline
Final \_ age
&
varchar(45)
&
utf8 \_ general \_ ci
&
YES
&&
Age (ka)
\\
\hline
Final \_ age \_ unc
&
varchar(45)
&
utf8 \_ general \_ ci
&
YES
&&
Age uncertainty (ka)
\\
\hline
Mineral \_ type
&
varchar(45)
&
utf8 \_ general \_ ci
&
YES
&&
Mineral type
\\
\hline
Grain \_ size
&
varchar(45)
&
utf8 \_ general \_ ci
&
YES
&&
Grain size (µm)
\\
\hline
Aliquot size
&
varchar(45)
&
utf8 \_ general \_ ci
&
YES
&&
Aliquot size (mg)
\\
\hline
Meas \_ protocol
&
longtext
&
utf8 \_ general \_ ci
&
YES
&&
Measurement protocol
\\
\hline
ESR \_ spectr \_ type
&
mediumtext
&
utf8 \_ general \_ ci
&
YES
&&
ESR spectrometre type
\\
\hline
Meas \_ temp
&
longtext
&
utf8 \_ general \_ ci
&
YES
&&
Measurement temperature (K)
\\
\hline
AAcquisition \_ parameters
&
mediumtext
&
utf8 \_ general \_ ci
&
YES
&&
Acquisition parameters
\\
\hline
Paramagnetic \_ centre
&
mediumtext
&
utf8 \_ general \_ ci
&
YES
&&
ESR signal
\\
\hline
Nb \_ Scans
&
varchar(45)
&
utf8 \_ general \_ ci
&
YES
&&
Number of scans
\\
\hline
Nb \_ repeat \_ meas
&
varchar(45)
&
utf8 \_ general \_ ci
&
YES
&&
Number of repeated measurements
\\
\hline
Nb \_ rotations
&
varchar(45)
&
utf8 \_ general \_ ci
&
YES
&&
Number of rotations in the cavity
\\
\hline
ESR \_ Sign \_ int
&
mediumtext
&
utf8 \_ general \_ ci
&
YES
&&
Evaluation of the ESR signal intensity
\\
\hline
Corr \_ ESR \_ int
&
mediumtext
&
utf8 \_ general \_ ci
&
YES
&&
Correction of the ESR intensities
\\
\hline
Bleaching \_ coeff
&
mediumtext
&
utf8 \_ general \_ ci
&
YES
&&
Bleaching coefficient (\%)
\\
\hline
Fitting \_ prog \_ err \_ calc
&
mediumtext
&
utf8 \_ general \_ ci
&
YES
&&
Fitting program and error calculation
\\
\hline
Fitting function
&
mediumtext
&
utf8 \_ general \_ ci
&
YES
&&
Fitting function
\\
\hline
Data \_ weight \_ fitting
&
mediumtext
&
utf8 \_ general \_ ci
&
YES
&&
Data weighting used for fitting
\\
\hline
Equivalent \_ dose
&
varchar(45)
&
utf8 \_ general \_ ci
&
YES
&&
Equivalent dose estimate (Gy)
\\
\hline
Equivalent \_ dose \_ err
&
varchar(45)
&
utf8 \_ general \_ ci
&
YES
&&
Equivalent dose estimate undertainty (\(\pm\)1sigma) (Gy)
\\
\hline
Depth
&
varchar(45)
&
utf8 \_ general \_ ci
&
YES
&&
Depth (m)
\\
\hline
Technique
&
longtext
&
utf8 \_ general \_ ci
&
YES
&&
Technique
\\
\hline
Dose \_ rate \_ unc \_ type
&
varchar(45)
&
utf8 \_ general \_ ci
&
YES
&&
Dose rate uncertainty type
\\
\hline
extU
&
varchar(45)
&
utf8 \_ general \_ ci
&
YES
&&
external U (ppm)
\\
\hline
extU \_ unc
&
varchar(45)
&
utf8 \_ general \_ ci
&
YES
&&
external U error (ppm)
\\
\hline
extTh
&
varchar(45)
&
utf8 \_ general \_ ci
&
YES
&&
external Th (ppm)
\\
\hline
extTh \_ unc
&
varchar(45)
&
utf8 \_ general \_ ci
&
YES
&&
external Th error (ppm)
\\
\hline
extK
&
varchar(45)
&
utf8 \_ general \_ ci
&
YES
&&
external K (\%)
\\
\hline
extK \_ unc
&
varchar(45)
&
utf8 \_ general \_ ci
&
YES
&&
external K error (\%)
\\
\hline
extRb
&
varchar(45)
&
utf8 \_ general \_ ci
&
YES
&&
external Rb (ppm)
\\
\hline
extRb \_ unc
&
varchar(45)
&
utf8 \_ general \_ ci
&
YES
&&
external Rb error (ppm)
\\
\hline
Ext \_ dose \_ rate
&
varchar(45)
&
utf8 \_ general \_ ci
&
YES
&&
External dose rate (Gy/ka)
\\
\hline
Ext \_ dose \_ rate \_ unc
&
varchar(45)
&
utf8 \_ general \_ ci
&
YES
&&
External dose rate uncertainty (Gy/ka)
\\
\hline
intU
&
varchar(45)
&
utf8 \_ general \_ ci
&
YES
&&
internal U (ppm)
\\
\hline
intU \_ unc
&
varchar(45)
&
utf8 \_ general \_ ci
&
YES
&&
internal U error (ppm)
\\
\hline
intTh
&
varchar(45)
&
utf8 \_ general \_ ci
&
YES
&&
internal Th (ppm)
\\
\hline
intTh \_ unc
&
varchar(45)
&
utf8 \_ general \_ ci
&
YES
&&
internal Th error (ppm)
\\
\hline
intK
&
varchar(45)
&
utf8 \_ general \_ ci
&
YES
&&
internal K (\%)
\\
\hline
intK \_ unc
&
varchar(45)
&
utf8 \_ general \_ ci
&
YES
&&
internal K error (\%)
\\
\hline
intRb
&
varchar(45)
&
utf8 \_ general \_ ci
&
YES
&&
internal Rb (ppm)
\\
\hline
intRb \_ unc
&
varchar(45)
&
utf8 \_ general \_ ci
&
YES
&&
internal Rb error (ppm)
\\
\hline
Int \_ dose \_ rate
&
varchar(45)
&
utf8 \_ general \_ ci
&
YES
&&
Internal dose rate (Gy/ka)
\\
\hline
Int \_ dose \_ rate \_ unc
&
varchar(45)
&
utf8 \_ general \_ ci
&
YES
&&
Internal dose rate uncertainty (Gy/ka)
\\
\hline
Alpha \_ rate \_ val
&
varchar(45)
&
utf8 \_ general \_ ci
&
YES
&&
Alpha dose rate (Gy/ka)
\\
\hline
Alpha \_ rate \_ unc
&
varchar(45)
&
utf8 \_ general \_ ci
&
YES
&&
Alpha dose rate error (Gy/ka)
\\
\hline
Gamma \_ rate \_ val
&
varchar(45)
&
utf8 \_ general \_ ci
&
YES
&&
Gamma dose rate (Gy/ka)
\\
\hline
Gamma \_ rate \_ unc
&
varchar(45)
&
utf8 \_ general \_ ci
&
YES
&&
Gamma dose rate error (Gy/ka)
\\
\hline
Beta \_ rate \_ val
&
varchar(45)
&
utf8 \_ general \_ ci
&
YES
&&
Beta dose rate (Gy/ka)
\\
\hline
Beta \_ rate \_ unc
&
varchar(45)
&
utf8 \_ general \_ ci
&
YES
&&
Beta dose rate error (Gy/ka)
\\
\hline
Cosmic \_ rate \_ val
&
varchar(45)
&
utf8 \_ general \_ ci
&
YES
&&
Cosmic dose rate (Gy/ka)
\\
\hline
Cosmic \_ rate \_ unc
&
varchar(45)
&
utf8 \_ general \_ ci
&
YES
&&
Cosmic dose rate error (Gy/ka)
\\
\hline
Field \_ water
&
varchar(45)
&
utf8 \_ general \_ ci
&
YES
&&
Field water content (\% of dry mass)
\\
\hline
Water \_ cont \_ det
&
longtext
&
utf8 \_ general \_ ci
&
YES
&&
Water content determination method
\\
\hline
Water \_ cont \_ val
&
varchar(45)
&
utf8 \_ general \_ ci
&
YES
&&
Water content used (\%)
\\
\hline
Water \_ cont \_ unc
&
varchar(45)
&
utf8 \_ general \_ ci
&
YES
&&
Water content used error (\%)
\\
\hline
Atten \_ fact
&
longtext
&
utf8 \_ general \_ ci
&
YES
&&
Attenuation factors
\\
\hline
Conv \_ fact
&
longtext
&
utf8 \_ general \_ ci
&
YES
&&
Conversion factors
\\
\hline
Alpha \_ eff \_ fact
&
longtext
&
utf8 \_ general \_ ci
&
YES
&&
Alpha efficiency factor
\\
\hline
Etching \_ method
&
longtext
&
utf8 \_ general \_ ci
&
YES
&&
Etching method
\\
\hline
etch \_ depth
&
varchar(45)
&
utf8 \_ general \_ ci
&
YES
&&
Etch depth (µm)
\\
\hline
etch \_ depth \_ unc
&
varchar(45)
&
utf8 \_ general \_ ci
&
YES
&&
Etch depth error (µm)
\\
\hline
Total \_ dose \_ rate
&
varchar(45)
&
utf8 \_ general \_ ci
&
YES
&&
Total dose rate (Gy/ka)
\\
\hline
Total \_ dose \_ rate \_ unc
&
varchar(45)
&
utf8 \_ general \_ ci
&
YES
&&
Total dose rate uncertainty (Gy/ka)
\\
\hline
Notes
&
longtext
&
utf8 \_ general \_ ci
&
YES
&&
Comments and notes
\\
\hline
Createdby
&
varchar(45)
&
utf8 \_ general \_ ci
&
YES
&&
Record created by
\\
\hline
Updatedby
&
varchar(45)
&
utf8 \_ general \_ ci
&
YES
&&
Record updated by
\\
\hline
LastUpdate
&
datetime
&&
YES
&&
Last Update
\\
\hline
coord
&
point
&&
YES
&&
mySQL coordinates
\\
\hline
\end{longtable}\sphinxatlongtableend\end{savenotes}


\section{Table: luminescence}
\label{\detokenize{database:table-luminescence}}
This table contains samples dated with luminescence techniques.


\begin{savenotes}\sphinxatlongtablestart\begin{longtable}[c]{|\X{20}{105}|\X{15}{105}|\X{25}{105}|\X{15}{105}|\X{10}{105}|\X{20}{105}|}
\hline
\sphinxstyletheadfamily 
Field
&\sphinxstyletheadfamily 
Type
&\sphinxstyletheadfamily 
Collation
&\sphinxstyletheadfamily 
Null
&\sphinxstyletheadfamily 
Key
&\sphinxstyletheadfamily 
Comment
\\
\hline
\endfirsthead

\multicolumn{6}{c}%
{\makebox[0pt]{\sphinxtablecontinued{\tablename\ \thetable{} \textendash{} continued from previous page}}}\\
\hline
\sphinxstyletheadfamily 
Field
&\sphinxstyletheadfamily 
Type
&\sphinxstyletheadfamily 
Collation
&\sphinxstyletheadfamily 
Null
&\sphinxstyletheadfamily 
Key
&\sphinxstyletheadfamily 
Comment
\\
\hline
\endhead

\hline
\multicolumn{6}{r}{\makebox[0pt][r]{\sphinxtablecontinued{continues on next page}}}\\
\endfoot

\endlastfoot

LUM \_ ID
&
int(11)
&&
NO
&
PRI
&
WALIS LUM ID
\\
\hline
Public
&
varchar(45)
&
utf8 \_ general \_ ci
&
YES
&&
Is this datapoint public?
\\
\hline
Lum \_ Method
&
tinytext
&
utf8 \_ general \_ ci
&
YES
&&
Luminescence method
\\
\hline
IGSN
&
varchar(45)
&
utf8 \_ general \_ ci
&
YES
&&
IGSN
\\
\hline
lum \_ ref
&
longtext
&
utf8 \_ general \_ ci
&
YES
&&
Reference (s)
\\
\hline
Sample \_ ID
&
varchar(45)
&
utf8 \_ general \_ ci
&
YES
&&
Sample ID
\\
\hline
AnalysisID
&
varchar(45)
&
utf8 \_ general \_ ci
&
YES
&&
Analysis ID
\\
\hline
ReportedID
&
varchar(45)
&
utf8 \_ general \_ ci
&
YES
&&
Reported ID
\\
\hline
Lab \_ name
&
mediumtext
&
utf8 \_ general \_ ci
&
YES
&&
Laboratory name
\\
\hline
Mineral \_ type
&
varchar(45)
&
utf8 \_ general \_ ci
&
YES
&&
Mineral type
\\
\hline
Grain \_ size
&
varchar(45)
&
utf8 \_ general \_ ci
&
YES
&&
Grain size
\\
\hline
AnalysisDate
&
datetime
&&
YES
&&
Date of analysis
\\
\hline
Accepted
&
varchar(45)
&
utf8 \_ general \_ ci
&
YES
&&
Accepted?
\\
\hline
Reas \_ rej
&
longtext
&
utf8 \_ general \_ ci
&
YES
&&
Reason for rejection
\\
\hline
Reported \_ Latitude
&
varchar(45)
&
utf8 \_ general \_ ci
&
YES
&&
Reported Latitude
\\
\hline
Reported \_ Longitude
&
varchar(45)
&
utf8 \_ general \_ ci
&
YES
&&
Reported Longitude
\\
\hline
Sample \_ lat
&
varchar(45)
&
utf8 \_ general \_ ci
&
YES
&&
Sample latitude (decimal degrees)
\\
\hline
Sample \_ lon
&
varchar(45)
&
utf8 \_ general \_ ci
&
YES
&&
Sample longitude (decimal degrees)
\\
\hline
Lat \_ Lon \_ Est
&
varchar(45)
&
utf8 \_ general \_ ci
&
YES
&&
Are Lat/Long estimated?
\\
\hline
Original \_ datum
&
varchar(45)
&
utf8 \_ general \_ ci
&
YES
&&
Original elevation datum used
\\
\hline
Elev \_ meas \_ method
&
varchar(45)
&
utf8 \_ general \_ ci
&
YES
&&
Elevation measurement method
\\
\hline
Sample \_ elev
&
varchar(45)
&
utf8 \_ general \_ ci
&
YES
&&
Reported elevation (m)
\\
\hline
Sample \_ elev \_ unc
&
varchar(45)
&
utf8 \_ general \_ ci
&
YES
&&
Reported elevation uncertainty (m)
\\
\hline
Elev \_ MSL
&
varchar(45)
&
utf8 \_ general \_ ci
&
YES
&&
Elevation above MSL (m)
\\
\hline
Elev \_ MSL \_ unc
&
varchar(45)
&
utf8 \_ general \_ ci
&
YES
&&
Elevation uncertainty used (m)
\\
\hline
Elev \_ comments
&
longtext
&
utf8 \_ general \_ ci
&
YES
&&
Comments on elevation
\\
\hline
MISchoice \_ limit
&
varchar(45)
&
utf8 \_ general \_ ci
&
YES
&&
Age is Older/Equal/Younger than
\\
\hline
MISchoice
&
varchar(45)
&
utf8 \_ general \_ ci
&
YES
&&
Marine Isotopic Stage
\\
\hline
Reported \_ MIS
&
longtext
&
utf8 \_ general \_ ci
&
YES
&&
Comments/details on MIS designation or age
\\
\hline
Final \_ age
&
varchar(45)
&
utf8 \_ general \_ ci
&
YES
&&
Age (ka)
\\
\hline
Final \_ age \_ unc
&
varchar(45)
&
utf8 \_ general \_ ci
&
YES
&&
Age uncertainty (ka)
\\
\hline
Aliquot size
&
varchar(45)
&
utf8 \_ general \_ ci
&
YES
&&
Aliquot size
\\
\hline
ED \_ meas \_ protocol
&
longtext
&
utf8 \_ general \_ ci
&
YES
&&
Equivalent Dose measurement protocol
\\
\hline
Treatment
&
longtext
&
utf8 \_ general \_ ci
&
YES
&&
Treatment during measurement
\\
\hline
ED \_ stat \_ model
&
varchar(45)
&
utf8 \_ general \_ ci
&
YES
&&
Statistical model for burial dose
\\
\hline
Machine \_ type
&
longtext
&
utf8 \_ general \_ ci
&
YES
&&
Machine type
\\
\hline
Detection \_ unit
&
longtext
&
utf8 \_ general \_ ci
&
YES
&&
Detection unit
\\
\hline
Stimulation \_ unit
&
longtext
&
utf8 \_ general \_ ci
&
YES
&&
Stimulation unit
\\
\hline
Optical \_ filters
&
longtext
&
utf8 \_ general \_ ci
&
YES
&&
Optical Filters used for Detection
\\
\hline
Nb \_ aliquots \_ sg \_ run
&
mediumtext
&
utf8 \_ general \_ ci
&
YES
&&
Number of aliquots/ single grains run
\\
\hline
Nb \_ aliquots \_ sg \_ used
&
mediumtext
&
utf8 \_ general \_ ci
&
YES
&&
Number of aliquots/ single grains used in sample
\\
\hline
Dose \_ unc \_ lev
&
varchar(45)
&
utf8 \_ general \_ ci
&
YES
&&
Uncertainty level
\\
\hline
ED
&
varchar(45)
&
utf8 \_ general \_ ci
&
YES
&&
Equivalent dose (Gy)
\\
\hline
ED \_ unc
&
varchar(45)
&
utf8 \_ general \_ ci
&
YES
&&
Equivalent dose uncertainty (Gy)
\\
\hline
OD \_ val
&
varchar(45)
&
utf8 \_ general \_ ci
&
YES
&&
Overdispersion value (\%)
\\
\hline
OD \_ val \_ unc
&
varchar(45)
&
utf8 \_ general \_ ci
&
YES
&&
Overdispersion value uncertainty (\%)
\\
\hline
Depth \_ of \_ burial
&
varchar(45)
&
utf8 \_ general \_ ci
&
YES
&&
Depth of burial (m)
\\
\hline
DR \_ determination
&
longtext
&
utf8 \_ general \_ ci
&
YES
&&
Approach of dose rate dermination
\\
\hline
Dose \_ rate \_ unc \_ type
&
varchar(45)
&
utf8 \_ general \_ ci
&
YES
&&
Uncertainty levels
\\
\hline
U
&
varchar(45)
&
utf8 \_ general \_ ci
&
YES
&&
Uranium content (ppm)
\\
\hline
U \_ unc
&
varchar(45)
&
utf8 \_ general \_ ci
&
YES
&&
Uranium content uncertainty (ppm)
\\
\hline
Th
&
varchar(45)
&
utf8 \_ general \_ ci
&
YES
&&
Thorium content (ppm)
\\
\hline
Th \_ unc
&
varchar(45)
&
utf8 \_ general \_ ci
&
YES
&&
Thorium content uncertainty (ppm)
\\
\hline
K
&
varchar(45)
&
utf8 \_ general \_ ci
&
YES
&&
Potassium content (\%)
\\
\hline
K \_ unc
&
varchar(45)
&
utf8 \_ general \_ ci
&
YES
&&
Potassium content uncertainty (\%)
\\
\hline
Rb
&
varchar(45)
&
utf8 \_ general \_ ci
&
YES
&&
Rubidium content (ppm)
\\
\hline
Rb \_ unc
&
varchar(45)
&
utf8 \_ general \_ ci
&
YES
&&
Rubidium content uncertainty (ppm)
\\
\hline
Alpha \_ rate \_ val
&
varchar(45)
&
utf8 \_ general \_ ci
&
YES
&&
Alpha dose rate (Gy/ka)
\\
\hline
Alpha \_ rate \_ unc
&
varchar(45)
&
utf8 \_ general \_ ci
&
YES
&&
Alpha dose rate uncertainty (Gy/ka)
\\
\hline
Gamma \_ rate \_ val
&
varchar(45)
&
utf8 \_ general \_ ci
&
YES
&&
Gamma dose rate (Gy/ka)
\\
\hline
Gamma \_ rate \_ unc
&
varchar(45)
&
utf8 \_ general \_ ci
&
YES
&&
Gamma dose rate uncertainty (Gy/ka)
\\
\hline
Beta \_ rate \_ val
&
varchar(45)
&
utf8 \_ general \_ ci
&
YES
&&
Beta dose rate (Gy/ka)
\\
\hline
Beta \_ rate \_ unc
&
varchar(45)
&
utf8 \_ general \_ ci
&
YES
&&
Beta dose rate uncertainty (Gy/ka)
\\
\hline
Cosmic \_ rate \_ val
&
varchar(45)
&
utf8 \_ general \_ ci
&
YES
&&
Cosmic dose rate (Gy/ka)
\\
\hline
Cosmic \_ rate \_ unc
&
varchar(45)
&
utf8 \_ general \_ ci
&
YES
&&
Cosmic dose rate uncertainty (Gy/ka)
\\
\hline
Field \_ water
&
varchar(45)
&
utf8 \_ general \_ ci
&
YES
&&
Field water content (\% of dry mass)
\\
\hline
Water \_ cont \_ det
&
longtext
&
utf8 \_ general \_ ci
&
YES
&&
Water content determination method
\\
\hline
Water \_ cont \_ val
&
varchar(45)
&
utf8 \_ general \_ ci
&
YES
&&
Water Content used in Final Age Calculation (\% of dry mass)
\\
\hline
Water \_ cont \_ unc
&
varchar(45)
&
utf8 \_ general \_ ci
&
YES
&&
Water Content used in Final Age Calculation  uncertainty (\% of dry mass)
\\
\hline
Atten \_ fact
&
varchar(45)
&
utf8 \_ general \_ ci
&
YES
&&
Alpha attenuation factor
\\
\hline
DR \_ conv \_ fact
&
varchar(45)
&
utf8 \_ general \_ ci
&
YES
&&
Dose Rate conversion factors
\\
\hline
Total \_ dose \_ rate
&
varchar(45)
&
utf8 \_ general \_ ci
&
YES
&&
Total dose rate (Gy/Ka)
\\
\hline
Total \_ dose \_ rate \_ unc
&
varchar(45)
&
utf8 \_ general \_ ci
&
YES
&&
Total dose rate uncertainty (Gy/ka)
\\
\hline
Fading \_ corr
&
varchar(45)
&
utf8 \_ general \_ ci
&
YES
&&
Was fading correction applied?
\\
\hline
Method \_ fading
&
varchar(45)
&
utf8 \_ general \_ ci
&
YES
&&
Method of fading correction
\\
\hline
g\sphinxhyphen{}value
&
varchar(45)
&
utf8 \_ general \_ ci
&
YES
&&
g\sphinxhyphen{}value (\% per decade)
\\
\hline
g\sphinxhyphen{}value \_ unc
&
varchar(45)
&
utf8 \_ general \_ ci
&
YES
&&
g\sphinxhyphen{}value (\(\pm\)1sigma) (\% per decade)
\\
\hline
Notes
&
longtext
&
utf8 \_ general \_ ci
&
YES
&&
Comments
\\
\hline
Createdby
&
varchar(45)
&
utf8 \_ general \_ ci
&
YES
&&
Record created by
\\
\hline
Updatedby
&
varchar(45)
&
utf8 \_ general \_ ci
&
YES
&&
Record updated by
\\
\hline
LastUpdate
&
datetime
&&
YES
&&
Last Update
\\
\hline
coord
&
point
&&
YES
&&
MySQL coordinates
\\
\hline
\end{longtable}\sphinxatlongtableend\end{savenotes}


\section{Table: strat}
\label{\detokenize{database:table-strat}}
This table contains chronostratigraphic age constraints.


\begin{savenotes}\sphinxattablestart
\centering
\begin{tabular}[t]{|\X{20}{105}|\X{15}{105}|\X{25}{105}|\X{15}{105}|\X{10}{105}|\X{20}{105}|}
\hline
\sphinxstyletheadfamily 
Field
&\sphinxstyletheadfamily 
Type
&\sphinxstyletheadfamily 
Collation
&\sphinxstyletheadfamily 
Null
&\sphinxstyletheadfamily 
Key
&\sphinxstyletheadfamily 
Comment
\\
\hline
Strat \_ ID
&
int(11)
&&
NO
&
PRI
&
WALIS Strat ID
\\
\hline
Public
&
tinytext
&
utf8 \_ general \_ ci
&
YES
&&
Is this datapoint public?
\\
\hline
StratName
&
varchar(45)
&
utf8 \_ general \_ ci
&
YES
&&
Chronostratigraphy ID
\\
\hline
StratRef
&
mediumtext
&
utf8 \_ general \_ ci
&
YES
&&
Reference(s)
\\
\hline
StratDescr
&
longtext
&
utf8 \_ general \_ ci
&
YES
&&
Description for chronostratigraphic constraint
\\
\hline
UnitThickness
&
mediumtext
&
utf8 \_ general \_ ci
&
YES
&&
Unit thickness (m)
\\
\hline
MISchoice \_ limits
&
varchar(45)
&
utf8 \_ general \_ ci
&
YES
&&
Age is Older/Equal/Younger than
\\
\hline
MISchoice
&
varchar(45)
&
utf8 \_ general \_ ci
&
YES
&&
Marine Isotopic Stage
\\
\hline
MIS
&
longtext
&
utf8 \_ general \_ ci
&
YES
&&
Comments/details on MIS designation
\\
\hline
UpperAge
&
decimal(10,0)
&&
YES
&&
Upper Age (ka)
\\
\hline
LowerAge
&
decimal(10,0)
&&
YES
&&
Lower Age (ka)
\\
\hline
AgeNotes
&
longtext
&
utf8 \_ general \_ ci
&
YES
&&
Notes on age determination
\\
\hline
OfficialStrat
&
varchar(45)
&
utf8 \_ general \_ ci
&
YES
&&
Is this an official stratigraphic designation?
\\
\hline
dur \_ yrs \_ y \_ n
&
varchar(45)
&
utf8 \_ general \_ ci
&
YES
&&
Has duration estimated in years?
\\
\hline
duration
&
varchar(45)
&
utf8 \_ general \_ ci
&
YES
&&
Duration (yrs)
\\
\hline
duration \_ unc
&
varchar(45)
&
utf8 \_ general \_ ci
&
YES
&&
Duration uncertainty (yrs)
\\
\hline
parent \_ strat \_ y \_ n
&
varchar(45)
&
utf8 \_ general \_ ci
&
YES
&&
Is a subzone of a parent Chronostratigraphy entry?
\\
\hline
parent \_ link
&
varchar(45)
&
utf8 \_ general \_ ci
&
YES
&&
Parent record (WALIS Strat ID)
\\
\hline
Createdby
&
varchar(45)
&
utf8 \_ general \_ ci
&
YES
&&
Record created by
\\
\hline
Updatedby
&
varchar(45)
&
utf8 \_ general \_ ci
&
YES
&&
Record updated by
\\
\hline
LastUpdate
&
datetime
&&
YES
&&
Last Update
\\
\hline
\end{tabular}
\par
\sphinxattableend\end{savenotes}


\section{Table: other\_dating}
\label{\detokenize{database:table-other-dating}}
This table contains samples dated with age techniques not detailed in WALIS.


\begin{savenotes}\sphinxattablestart
\centering
\begin{tabular}[t]{|\X{20}{105}|\X{15}{105}|\X{25}{105}|\X{15}{105}|\X{10}{105}|\X{20}{105}|}
\hline
\sphinxstyletheadfamily 
Field
&\sphinxstyletheadfamily 
Type
&\sphinxstyletheadfamily 
Collation
&\sphinxstyletheadfamily 
Null
&\sphinxstyletheadfamily 
Key
&\sphinxstyletheadfamily 
Comment
\\
\hline
idOther \_ dating
&
int(11)
&&
NO
&
PRI
&
WALIS Other chronology ID
\\
\hline
Public
&
varchar(45)
&
utf8 \_ general \_ ci
&
YES
&&
Is this datapoint public?
\\
\hline
Short name
&
varchar(45)
&
utf8 \_ general \_ ci
&
YES
&&
Short name
\\
\hline
Dating \_ technique
&
mediumtext
&
utf8 \_ general \_ ci
&
YES
&&
Dating technique
\\
\hline
Description
&
longtext
&
utf8 \_ general \_ ci
&
YES
&&
Description
\\
\hline
Upper age
&
decimal(10,2)
&&
YES
&&
Upper age (ka)
\\
\hline
Lower age
&
decimal(10,2)
&&
YES
&&
Lower age (ka)
\\
\hline
MISchoice \_ limit
&
varchar(45)
&
utf8 \_ general \_ ci
&
YES
&&
Age is Older/Equal/Younger than
\\
\hline
MISchoice
&
varchar(45)
&
utf8 \_ general \_ ci
&
YES
&&
Marine Isotopic Stage
\\
\hline
MIS
&
longtext
&
utf8 \_ general \_ ci
&
YES
&&
Comments/details on MIS designation
\\
\hline
Ref
&
varchar(45)
&
utf8 \_ general \_ ci
&
YES
&&
Reference(s)
\\
\hline
Createdby
&
varchar(45)
&
utf8 \_ general \_ ci
&
YES
&&
Record created by
\\
\hline
Updatedby
&
varchar(45)
&
utf8 \_ general \_ ci
&
YES
&&
Record updated by
\\
\hline
LastUpdate
&
datetime
&&
YES
&&
Last Update
\\
\hline
\end{tabular}
\par
\sphinxattableend\end{savenotes}


\chapter{Database interface}
\label{\detokenize{interface:database-interface}}\label{\detokenize{interface::doc}}
The compilation interface for sea\sphinxhyphen{}level datapoints gives access to a sub\sphinxhyphen{}menu with three options:
\begin{enumerate}
\sphinxsetlistlabels{\arabic}{enumi}{enumii}{}{.}%
\item {} 
RSL datapoint from stratigraphy

\item {} 
RSL datapoint from single coral

\item {} 
RSL datapoint from speleothem

\end{enumerate}
\begin{quote}

\begin{figure}[htbp]
\centering
\capstart

\noindent\sphinxincludegraphics[width=1.000\linewidth]{{SL_datapoints}.png}
\caption{Screenshot of the ‘Sea level datapoints’ submenu.}\label{\detokenize{interface:id1}}\end{figure}
\end{quote}


\section{RSL datapoint from stratigraphy}
\label{\detokenize{interface:rsl-datapoint-from-stratigraphy}}
The ‘RSL datapoint from stratigraphy’ sub\sphinxhyphen{}menu compiles the table ‘rsl’ in the mySQL database. In the interface, the compilation is divided into five tabs.

\begin{sphinxadmonition}{note}{Note:}
This section has been designed by A. Rovere and D. Ryan
\end{sphinxadmonition}

\begin{sphinxadmonition}{tip}{Tip:}
This sub\sphinxhyphen{}menu should be used if geomorphological or stratigraphic information to reconstruct paleo RSL is available
\end{sphinxadmonition}

\begin{figure}[htbp]
\centering
\capstart

\noindent\sphinxincludegraphics[width=1.000\linewidth]{{RSL_datapoints_strat}.png}
\caption{Screen shot of the ‘RSL datapoint from stratigraphy’ interface}\label{\detokenize{interface:id2}}\end{figure}


\subsection{Interface versions}
\label{\detokenize{interface:interface-versions}}
The following table shows the main changes corresponding with each interface versions


\begin{savenotes}\sphinxatlongtablestart\begin{longtable}[c]{|\X{5}{23}|\X{10}{23}|\X{8}{23}|}
\hline
\sphinxstyletheadfamily 
Version
&\sphinxstyletheadfamily 
Changes
&\sphinxstyletheadfamily 
Date
\\
\hline
\endfirsthead

\multicolumn{3}{c}%
{\makebox[0pt]{\sphinxtablecontinued{\tablename\ \thetable{} \textendash{} continued from previous page}}}\\
\hline
\sphinxstyletheadfamily 
Version
&\sphinxstyletheadfamily 
Changes
&\sphinxstyletheadfamily 
Date
\\
\hline
\endhead

\hline
\multicolumn{3}{r}{\makebox[0pt][r]{\sphinxtablecontinued{continues on next page}}}\\
\endfoot

\endlastfoot
&&\\
\hline
1.7.0
&
First inclusion of Holocene data
&
02/16/21  18:19
\\
\hline&&\\
\hline
1.6.3
&
Small updates to interface
&
11/09/20  15:23
\\
\hline&&\\
\hline
1.6.2
&
Shift to AWS hosting
&
05/09/20  15:07
\\
\hline&&\\
\hline
1.6.1
&
General improvements
&
04/23/20  16:26
\\
\hline&&\\
\hline
1.6.0
&
Inserted python download functions
&
04/15/20  13:54
\\
\hline&&\\
\hline
1.5.6
&
New security system with logged users
&
04/15/20  12:32
\\
\hline&&\\
\hline
1.5.4
&
Minor updates and bug fixes
&
04/09/20  14:20
\\
\hline&&\\
\hline
1.5.3
&
Minor updates and bug fixes
&
03/16/20  10:24
\\
\hline&&\\
\hline
1.5.2
&
Minor updates and bug fixes
&
02/29/20  22:56
\\
\hline&&\\
\hline
1.5.1
&
Revised maps
&
02/03/20  15:20
\\
\hline&&\\
\hline
1.5.0
&
Heavily revised table reports and maps
&
01/26/20  13:09
\\
\hline&&\\
\hline
1.4.5
&
Added map of user sites
&
01/21/20  09:36
\\
\hline&&\\
\hline
1.4.4
&
Fixed bugs on different tables and started to improve help functions
&
01/20/20  17:21
\\
\hline&&\\
\hline
1.4.3
&
Changes in Luminescence and updates of search function
&
01/15/20  16:01
\\
\hline&&\\
\hline
1.4.2
&
Slight changes to ESR and Chronostratigrapy
&
01/08/20  11:27
\\
\hline&&\\
\hline
1.4.0
&
Changes in database structure for ESR, AAR and LUM. Other minor changes on the layout.
&
10/18/19  11:53
\\
\hline&&\\
\hline
1.3.0
&
Minor style and messages updates
&
09/11/19  17:00
\\
\hline&&\\
\hline
1.2.7
&
Minor style and messages updates
&
09/10/19  21:52
\\
\hline&&\\
\hline
1.2.6
&
Minor app and database bug fixes
&
08/19/19  19:25
\\
\hline&&\\
\hline
1.2.5
&
Bug fixes and updates
&
07/28/19  14:58
\\
\hline&&\\
\hline
1.2.4
&
Bug fixes and updates
&
07/28/19  11:46
\\
\hline&&\\
\hline
1.2.3
&
Bug fixes and updates
&
07/24/19  11:38
\\
\hline&&\\
\hline
1.2.2
&
Bug fixes and updates
&
07/10/19  18:37
\\
\hline&&\\
\hline
1.2.1
&
Bug fixes and updates
&
06/19/19  11:03
\\
\hline&&\\
\hline
1.2.0
&
Bug fixes and updates
&
06/06/19  15:28
\\
\hline&&\\
\hline
1.1.5
&
First online version
&
06/03/19  19:59
\\
\hline&&\\
\hline
1.1.4
&
Small fixes
&
05/26/19  13:36
\\
\hline&&\\
\hline
1.1.3
&
Small fixes
&
05/20/19  16:12
\\
\hline&&\\
\hline
1.1.2
&
Built menus and grids
&
05/19/19  11:46
\\
\hline&&\\
\hline
1.1.1
&
Built forms
&
05/04/19  15:45
\\
\hline&&\\
\hline
1.0.0
&
Initial beta
&
12/18/18  17:58
\\
\hline&&\\
\hline
\end{longtable}\sphinxatlongtableend\end{savenotes}


\chapter{RSL from stratigraphy}
\label{\detokenize{RSL_data:rsl-from-stratigraphy}}\label{\detokenize{RSL_data::doc}}
\begin{sphinxadmonition}{note}{Note:}
This section has been designed by A. Rovere and D. Ryan
\end{sphinxadmonition}

This section describes the fields included in the ‘RSL proxies’ file/sheet, created from the ‘rsl’ mySQL table. In the interface, the fields below can be found under ‘RSL datapoint from stratigraphy’ page.


\section{General information}
\label{\detokenize{RSL_data:general-information}}
\sphinxstylestrong{Site name} \sphinxhyphen{} Name of the site. This could be a local geographic name, or the name of the town/city where the site is located.

\sphinxstylestrong{Subsite} \sphinxhyphen{} Survey name or transect number assigned to the site. This may also include a sub\sphinxhyphen{}facies within the sea level sequence.

\begin{sphinxadmonition}{note}{Note:}
In the interface, the combination of Site and Subsite is set as ‘Unique key’. This means that if a user tries to insert an existing Site\sphinxhyphen{}Subsite combination, it will not be accepted as valid by the interface and the record will not be saved. The following message will appear

\noindent{\hspace*{\fill}\sphinxincludegraphics[width=0.500\linewidth]{{error_site_subsite}.png}\hspace*{\fill}}
\end{sphinxadmonition}

\sphinxstylestrong{Nation} \sphinxhyphen{} The nation where the site is located.

\begin{sphinxadmonition}{warning}{Warning:}
The value for this field is selected from the ‘countries’ table. No foreign keys are defined in the database. If the required nation is not listed in the interface, please leave a bug report.
\end{sphinxadmonition}

\sphinxstylestrong{Region} \sphinxhyphen{} The administrative region, province or administrative unit where the site is located.

\begin{sphinxadmonition}{warning}{Warning:}
The value for this field is selected from the ‘regions’ table. No foreign keys are defined in the database. If the required region/administrative unit is not listed in the interface, please leave a bug report.
\end{sphinxadmonition}

\begin{sphinxadmonition}{note}{Note:}
Nations and Regions derived from a simplified version of this \sphinxhref{https://github.com/prograhammer/countries-regions-cities}{database}, which includes data created by MaxMind, available from \sphinxurl{http://www.maxmind.com/}. Refer to these links for further use and/or redistribution
\end{sphinxadmonition}

\sphinxstylestrong{Main reference} \sphinxhyphen{} The main reference from which RSL information has been extracted. This is usually the reference describing the site and reporting measurement and dating details.

\begin{sphinxadmonition}{warning}{Warning:}
The value for this field is selected from the ‘references’ table. It is set as foreign key in the ‘rsl’ table.
\end{sphinxadmonition}

\sphinxstylestrong{Additional References} \sphinxhyphen{} Select further references describing the site. Ideally this field includes all the historical references reporting on the site.

\begin{sphinxadmonition}{warning}{Warning:}
The values for this field are selected from the ‘references’ table. The ID values are stored as comma\sphinxhyphen{}separated. No foreign keys are defined in the database.
\end{sphinxadmonition}

\begin{sphinxadmonition}{note}{Note:}
The initial set of paleo\sphinxhyphen{}sea level references was derived from the works of \sphinxhref{https://doi.org/10.1016/j.earscirev.2014.01.007}{Pedoja et al (2014)}, \sphinxhref{https://doi.org/10.1016/j.quascirev.2016.04.019}{Hibbert et al (2016)}, and \sphinxhref{https://doi.org/10.1016/j.quaint.2005.07.009}{Ferranti et al (2006)}
\end{sphinxadmonition}

\sphinxstylestrong{Latitude (decimal degrees)} \sphinxhyphen{} Latitude of the site, in decimal degrees.

\sphinxstylestrong{Longitude (decimal degrees)} \sphinxhyphen{} Longitude of the site, in decimal degrees.

\sphinxstylestrong{Horizontal Positioning Technique} \sphinxhyphen{} Define the positioning technique used to establish the site Lat/Long coordinates. New positioning techniques can be added as necessary within the ‘Metadata \sphinxhyphen{}\textgreater{} Geographic positioning’ menu.

\begin{sphinxadmonition}{warning}{Warning:}
The value for this field is selected from the ‘hrzpostech’ table. No foreign keys are defined in the database.
\end{sphinxadmonition}

\sphinxstylestrong{Is this a marine/terrestrial limiting record?} \sphinxhyphen{} Indicate if the record inserted is: 1) marine or 2) terrestrial limiting, or 3) sea level indicator. The following fields will update accordingly. For a definition of marine/terrestrial limiting in MIS 5e, see \sphinxhref{https://doi.org/10.1016/j.earscirev.2016.06.006}{Rovere et al (2016)}.

\begin{sphinxadmonition}{note}{Note:}
In the interface, if (marine or terrestrial) limiting point is selected, a message appears: \sphinxstyleemphasis{Marine or Terrestial limiting, no paleo RSL calculations are possible} Contextually, the field \sphinxstyleemphasis{Type of RSL indicator} is hidden and associated value = 0, that corresponds to marine or terrestrial limiting in the ‘rsl\_ind’ table. Also the following fields are hidden: \sphinxstyleemphasis{Upper limit of modern analog (m)},*Lower limit of modern analog (m)*, \sphinxstyleemphasis{Quantification of indicative meaning},*Reference Water Level (m)*, \sphinxstyleemphasis{Indicative Range (m)},*Paleo Relative Sea Level (m)*, \sphinxstyleemphasis{Paleo Relative Sea Level Uncertainty (m)}. If sea level index point is selected, all the fields above are shown and made mandatory.
\end{sphinxadmonition}

\sphinxstylestrong{Type of RSL Indicator} \sphinxhyphen{} Sea level indicator that was reported at the site. New sea level indicators can be added as necessary within the ‘Metadata \sphinxhyphen{}\textgreater{} Type of RSL indicator’ menu.

\begin{sphinxadmonition}{warning}{Warning:}
The value for this field is selected from the ‘rsl\_ind’ table. No foreign keys are defined in the database.
\end{sphinxadmonition}

\begin{sphinxadmonition}{warning}{Warning:}
In the interface, this field appears only if ‘Sea Level indicator’ is selected in the field \sphinxstyleemphasis{‘Is this a marine/terrestrial limiting record?’}
\end{sphinxadmonition}

\sphinxstylestrong{Indicator description} \sphinxhyphen{} Describe the indicator/limiting point and its site\sphinxhyphen{}specific properties.


\section{Elevation and paleo RSL}
\label{\detokenize{RSL_data:elevation-and-paleo-rsl}}
\sphinxstylestrong{Upper limit of modern analog (m)} \sphinxhyphen{} Elevation of the highest point at which the facies/landform occurs along the modern shoreline.

\begin{sphinxadmonition}{warning}{Warning:}
In the interface, this field appears only if ‘Sea Level indicator’ is selected in the field \sphinxstyleemphasis{‘Is this a marine/terrestrial limiting record?’}
\end{sphinxadmonition}

\sphinxstylestrong{Lower limit of modern analog (m)} \sphinxhyphen{} Elevation of the lowest point at which the facies/landform occurs along the modern shoreline.

\begin{sphinxadmonition}{warning}{Warning:}
In the interface, this field appears only if ‘Sea Level indicator’ is selected in the field \sphinxstyleemphasis{‘Is this a marine/terrestrial limiting record?’}
\end{sphinxadmonition}

\sphinxstylestrong{Quantification of Indicative meaning} \sphinxhyphen{} Indicate how the indicative meaning has been quantified. Two selections are possible:
\begin{enumerate}
\sphinxsetlistlabels{\arabic}{enumi}{enumii}{}{.}%
\item {} 
IMCalc \sphinxhyphen{} Using the tool developed by \sphinxhref{https://doi.org/10.1186/s40965-019-0069-8}{Lorscheid and Rovere (2019)}

\item {} 
Modern analog data \sphinxhyphen{} From modern analog data available for the study area. See example in \sphinxhref{https://doi.org/10.1016/j.earscirev.2016.06.006}{Rovere et al (2016)}

\end{enumerate}

\begin{sphinxadmonition}{tip}{Tip:}
IMCalc is open access, runs on Java and can be downloaded from \sphinxhref{https://sourceforge.net/projects/imcalc/}{SourceForge}
\end{sphinxadmonition}

\begin{sphinxadmonition}{warning}{Warning:}
In the interface, this field appears only if ‘Sea Level indicator’ is selected in the field \sphinxstyleemphasis{‘Is this a marine/terrestrial limiting record?’}
\end{sphinxadmonition}

\sphinxstylestrong{Sea level datum} \sphinxhyphen{} Vertical datum to which the elevations are referred.

\begin{sphinxadmonition}{warning}{Warning:}
The value for this field is selected from the ‘sldatum’ table. No foreign keys are defined in the database.
\end{sphinxadmonition}

\sphinxstylestrong{Elevation measurement technique} \sphinxhyphen{} Method employed to measure elevations.

\begin{sphinxadmonition}{warning}{Warning:}
The value for this field is selected from the ‘vertmeastech’ table. No foreign keys are defined in the database.
\end{sphinxadmonition}

\sphinxstylestrong{Do you want to insert upper and lower elevation limits?} \sphinxhyphen{} Select ‘Yes’ if the sea level indicator was measured as two\sphinxhyphen{}points elevation. ‘No’ to insert only a single point elevation. See illustration below.

\begin{figure}[htbp]
\centering
\capstart

\noindent\sphinxincludegraphics[width=0.650\linewidth]{{Limits}.jpg}
\caption{Difference between one\sphinxhyphen{}point versus two\sphinxhyphen{}point elevation}\label{\detokenize{RSL_data:id6}}\end{figure}

\begin{sphinxadmonition}{note}{Note:}
In the interface, if ‘Yes’ is selected, the fields \sphinxstyleemphasis{Upper elevation of indicator (m)}, \sphinxstyleemphasis{Lower elevation of indicator (m)} and \sphinxstyleemphasis{Upper/Lower elevation measurement error (m)}  appear. The fields \sphinxstyleemphasis{RSL indicator elevation (m)} and \sphinxstyleemphasis{RSL indicator elevation error (m)} are disabled and calculated automatically averaging the values above. Every time the answer is changed, the content of the fields \sphinxstyleemphasis{RSL indicator elevation (m)} and \sphinxstyleemphasis{RSL indicator elevation error (m)} is deleted.
\end{sphinxadmonition}

\sphinxstylestrong{Lower elevation of indicator} \sphinxhyphen{} The lower measured elevation of the sea level indicator.

\begin{sphinxadmonition}{warning}{Warning:}
In the interface, this field appears and is mandatory only if ‘Yes’ is selected in the field \sphinxstyleemphasis{‘Do you want to insert upper and lower elevation limits?’}
\end{sphinxadmonition}

\sphinxstylestrong{Upper elevation of indicator} \sphinxhyphen{} The upper measured elevation of the sea level indicator.

\begin{sphinxadmonition}{warning}{Warning:}
In the interface, this field appears and is mandatory only if ‘Yes’ is selected in the field \sphinxstyleemphasis{‘Do you want to insert upper and lower elevation limits?’}
\end{sphinxadmonition}

\sphinxstylestrong{Upper/Lower elevation measurement error (m)} \sphinxhyphen{} Insert here the elevation measurement error for the Upper and Lower elevation measurements. Insert \(\pm\)1\sphinxhyphen{}sigma values.

\begin{sphinxadmonition}{warning}{Warning:}
In the interface, this field appears and is mandatory only if ‘Yes’ is selected in the field \sphinxstyleemphasis{‘Do you want to insert upper and lower elevation limits?’}
\end{sphinxadmonition}

\sphinxstylestrong{RSL indicator elevation (m)} \sphinxhyphen{} Elevation of the RSL indicator, in meters and with up to two decimal digits.

\begin{sphinxadmonition}{note}{Note:}
In the interface, this value is auto\sphinxhyphen{}calculated if upper and lower elevation limits are specified above.
\end{sphinxadmonition}

\sphinxstylestrong{RSL indicator elevation error (m)} \sphinxhyphen{} Elevation error associated with the elevation measurement, in meters and with up to two decimal digits. For GPS and similar instrumental measurement, use \(\pm\)1 sigma error levels.

\begin{sphinxadmonition}{note}{Note:}
In the interface, this value is auto\sphinxhyphen{}calculated if upper and lower elevation limits are specified above.
\end{sphinxadmonition}

\sphinxstylestrong{Notes on elevation} \sphinxhyphen{} Insert comments on how elevation data has been obtained / treated.

\sphinxstylestrong{Reference Water Level (m)} \sphinxhyphen{} The reference water level is calculated from the Upper and lower limits of modern analog inserted above, from the formulas in in \sphinxhref{https://doi.org/10.1016/j.earscirev.2016.06.006}{Rovere et al (2016)}.

\begin{sphinxadmonition}{warning}{Warning:}
In the interface, this field appears and is mandatory only if ‘Sea Level indicator’ is selected in the field \sphinxstyleemphasis{‘Is this a marine/terrestrial limiting record?’}
\end{sphinxadmonition}

\sphinxstylestrong{Indicative Range (m)} \sphinxhyphen{} The indicative range is calculated from the Upper and lower limits of modern analog inserted above, from the formulas in in \sphinxhref{https://doi.org/10.1016/j.earscirev.2016.06.006}{Rovere et al (2016)}.

\begin{sphinxadmonition}{warning}{Warning:}
In the interface, this field appears and is mandatory   only if ‘Sea Level indicator’ is selected in the field \sphinxstyleemphasis{‘Is this a marine/terrestrial limiting record?’}
\end{sphinxadmonition}

\sphinxstylestrong{Paleo Relative Sea Level (m)} \sphinxhyphen{} Paleo Relative sea level is calculated from the elevation and reference water level, from the formulas in in \sphinxhref{https://doi.org/10.1016/j.earscirev.2016.06.006}{Rovere et al (2016)}.

\begin{sphinxadmonition}{warning}{Warning:}
In the interface, this field appears and is auto\sphinxhyphen{}calculated only if ‘Sea Level indicator’ is selected in the field \sphinxstyleemphasis{‘Is this a marine/terrestrial limiting record?’}
\end{sphinxadmonition}

\sphinxstylestrong{Paleo Relative Sea Level Uncertainty (m)} \sphinxhyphen{} Paleo Relative sea level uncertainty is calculated from the elevation error and indicative range, from the formulas in in \sphinxhref{https://doi.org/10.1016/j.earscirev.2016.06.006}{Rovere et al (2016)}.

\begin{sphinxadmonition}{warning}{Warning:}
In the interface, this field appears and is auto\sphinxhyphen{}calculated only if ‘Sea Level indicator’ is selected in the field \sphinxstyleemphasis{‘Is this a marine/terrestrial limiting record?’}
\end{sphinxadmonition}


\section{Vertical land movements}
\label{\detokenize{RSL_data:vertical-land-movements}}
\sphinxstylestrong{Is data on vertical land movements (independent from the sea level record) available?} \sphinxhyphen{} Select ‘Yes’ or ‘No’ depending on the available data.

\begin{sphinxadmonition}{tip}{Tip:}
Compile data on vertical land movements only if \sphinxstyleemphasis{independent} vertical land motions are available for the site. Tectonic rates characterized by circularity (e.g., derived calculating the departure from Last Interglacial global mean sea level) should not be inserted
\end{sphinxadmonition}

\sphinxstylestrong{Tectonic category} \sphinxhyphen{} Three values are available: ‘Stable’, ‘Uplifting’ and ‘Subsidence’.

\sphinxstylestrong{Comments on tectonic category} \sphinxhyphen{} Comments on the selection of the tectonic category.

\sphinxstylestrong{Published VLM rate (m/ky)} \sphinxhyphen{} Vertical land motion rate as originally reported.

\sphinxstylestrong{Published VLM rate uncertainty (m/ky)} \sphinxhyphen{} Vertical land motion uncertainty as originally reported. If not reported, insert N/A.

\sphinxstylestrong{Interpreted VLM rate (m/ky)} \sphinxhyphen{} Vertical land motion rate as re\sphinxhyphen{}interpreted by the reviewer.

\sphinxstylestrong{Interpreted VLM rate uncertainty (m/ky)} \sphinxhyphen{} Vertical land motion uncertainty, as re\sphinxhyphen{}interpreted by the reviewer. If not reported, insert N/A.

\sphinxstylestrong{Comments on VLM rates} \sphinxhyphen{} Description of how the VLM rates reported have been calculated or re\sphinxhyphen{}calculated.


\section{Age constraints}
\label{\detokenize{RSL_data:age-constraints}}
\sphinxstylestrong{Age attribution} \sphinxhyphen{} Insert in this field one or more methods of age attribution. In the database, ‘Age attribution’ is coded in the following way:


\begin{savenotes}\sphinxattablestart
\centering
\begin{tabular}[t]{|\X{15}{60}|\X{45}{60}|}
\hline
\sphinxstyletheadfamily 
Value
&\sphinxstyletheadfamily 
Associated dating technique
\\
\hline\sphinxstyletheadfamily 
1
&
U\sphinxhyphen{}Series
\\
\hline\sphinxstyletheadfamily 
2
&
Amino Acid Racemization
\\
\hline\sphinxstyletheadfamily 
3
&
Luminescence
\\
\hline\sphinxstyletheadfamily 
4
&
Electron Spin Resonance
\\
\hline\sphinxstyletheadfamily 
5
&
Stratigraphic context/age
\\
\hline\sphinxstyletheadfamily 
6
&
Other
\\
\hline
\end{tabular}
\par
\sphinxattableend\end{savenotes}

\begin{sphinxadmonition}{warning}{Warning:}
At least one age constraint should be associated with the datapoint
\end{sphinxadmonition}

\begin{sphinxadmonition}{note}{Note:}
In the interface, once an age attribution method is selected, fields related to it appear and are made mandatory.
\end{sphinxadmonition}

\sphinxstylestrong{U\sphinxhyphen{}Series constraint} \sphinxhyphen{} Indicate if the RSL datapoint is ‘Equal to’, ‘Older than’, ‘Younger than’ the U\sphinxhyphen{}Series age constraint(s) inserted below.

\sphinxstylestrong{U\sphinxhyphen{}Series age IDs} \sphinxhyphen{} Select the U\sphinxhyphen{}Series age IDs. The list is created from all the ages present in the ‘Useries\_Corals’ table.

\begin{sphinxadmonition}{note}{Note:}
‘Useries\_Corals’ contains \sphinxstylestrong{all} U\sphinxhyphen{}Series ages, on corals, mollusks and speleothems. The table name is a legacy from an older database version.
\end{sphinxadmonition}

\begin{sphinxadmonition}{warning}{Warning:}
The values for this field are selected from the ‘Useries\_Corals’ table. The ID values are stored as comma\sphinxhyphen{}separated. No foreign keys are defined in the database.
\end{sphinxadmonition}

\sphinxstylestrong{AAR constraint} \sphinxhyphen{} Indicate if the RSL datapoint is is ‘Equal to’, ‘Older than’, ‘Younger than’ the AAR age constraint(s) inserted below.

\sphinxstylestrong{AAR age IDs} \sphinxhyphen{} Select the AAR age IDs. The list is created from all the ages present in the ‘aar’ table.

\begin{sphinxadmonition}{warning}{Warning:}
The values for this field are selected from the ‘aar’ table. The ID values are stored as comma\sphinxhyphen{}separated. No foreign keys are defined in the database.
\end{sphinxadmonition}

\sphinxstylestrong{Luminescence constraint} \sphinxhyphen{} Indicate if the RSL datapoint is is ‘Equal to’, ‘Older than’, ‘Younger than’ the luminescence age constraint(s) inserted below.

\sphinxstylestrong{Luminescence age IDs} \sphinxhyphen{} Select the luminescence age IDs. The list is created from all the ages present in the ‘luminescence’ table.

\begin{sphinxadmonition}{warning}{Warning:}
The values for this field are selected from the ‘luminescence’ table. The ID values are stored as comma\sphinxhyphen{}separated. No foreign keys are defined in the database.
\end{sphinxadmonition}

\sphinxstylestrong{ESR constraint} \sphinxhyphen{} Indicate if the RSL datapoint is is ‘Equal to’, ‘Older than’, ‘Younger than’ the ESR age constraint(s) inserted below.

\sphinxstylestrong{ESR age IDs} \sphinxhyphen{} Select the ESR age IDs. The list is created from all the ages present in the ‘esr’ table.

\begin{sphinxadmonition}{warning}{Warning:}
The values for this field are selected from the ‘esr’ table. The ID values are stored as comma\sphinxhyphen{}separated. No foreign keys are defined in the database.
\end{sphinxadmonition}

\sphinxstylestrong{Stratigraphic constraint} \sphinxhyphen{} Indicate if the RSL datapoint is is ‘Equal to’, ‘Older than’, ‘Younger than’ the stratigraphic age constraint(s) inserted below.

\sphinxstylestrong{Stratigraphic age IDs} \sphinxhyphen{} Select the stratigraphic age IDs. The list is created from all the ages present in the ‘strat’ table.

\begin{sphinxadmonition}{warning}{Warning:}
The values for this field are selected from the ‘strat’ table. The ID values are stored as comma\sphinxhyphen{}separated. No foreign keys are defined in the database.
\end{sphinxadmonition}

\sphinxstylestrong{Other age constraint} \sphinxhyphen{} Indicate if the RSL datapoint is is ‘Equal to’, ‘Older than’, ‘Younger than’ the other age age constraint(s) inserted below.

\sphinxstylestrong{Other age IDs} \sphinxhyphen{} Select the stratigraphic age IDs. The list is created from all the ages present in the ‘other\_dating’ table.

\begin{sphinxadmonition}{warning}{Warning:}
The values for this field are selected from the ‘other\_dating’ table. The ID values are stored as comma\sphinxhyphen{}separated. No foreign keys are defined in the database.
\end{sphinxadmonition}


\section{Quality}
\label{\detokenize{RSL_data:quality}}
\sphinxstylestrong{Quality of RSL data} \sphinxhyphen{} Rate the quality of the record for which concents RSL data, on a 1 (poor) to 5 (excellent) scale. If the record is rejected, select 0 stars and insert the reason for rejection below. A guide on the evaluation of quality can be found below:


\begin{savenotes}\sphinxattablestart
\centering
\begin{tabulary}{\linewidth}[t]{|T|T|}
\hline

Description
&
Quality rating
\\
\hline
Elevation precisely measured, referred to a clear datum and RSL indicator with a very narrow indicative range. Final RSL uncertainty is submetric.
&
5 (excellent)
\\
\hline
Elevation precisely measured, referred to a clear datum and RSL indicator with a narrow indicative range. Final RSL uncertainty is between one and two meters.
&
4 (good)
\\
\hline
Uncertainties in elevation, datum or indicative range sum up to a value between two and three meters.
&
3 (average)
\\
\hline
Final paleo RSL uncertainty is higher than three meters
&
2 (poor)
\\
\hline
Elevation and / or indicative range must be regarded as very uncertain due to poor measurement / description / RSL indicator quality
&
1 (very poor)
\\
\hline
There is not enough information to accept the record as a valid RSL indicator (e.g. marine or terrestial limiting)
&
0 (rejected)
\\
\hline
\end{tabulary}
\par
\sphinxattableend\end{savenotes}

\sphinxstylestrong{Quality of age information} \sphinxhyphen{} Rate the quality of the record for which concerns age data, on a 1 (poor) to 5 (excellent) scale. If the record is rejected, select 0 stars and insert the reason for rejection below. A guide on the evaluation of quality can be found below:


\begin{savenotes}\sphinxattablestart
\centering
\begin{tabulary}{\linewidth}[t]{|T|T|}
\hline

Description
&
Quality rating
\\
\hline
Very narrow age range, e.g. few ka, that allow the attribution to a specific timing within a substage of MIS 5 (e.g. 117\(\pm\)2 ka)
&
5 (excellent)
\\
\hline
Narrow age range, allowing the attribution to a specific substage of MIS 5 (e.g., MIS 5e)
&
4 (good)
\\
\hline
The RSL data point can be attributed only to a generic interglacial (e.g. MIS 5)
&
3 (average)
\\
\hline
Only partial information or minimum age constraints are available
&
2 (poor)
\\
\hline
Different age constraints point to different interglacials
&
1 (very poor)
\\
\hline
Not enough information to attribute the RSL data point to any pleistocene interglacial.
&
0 (rejected)
\\
\hline
\end{tabulary}
\par
\sphinxattableend\end{savenotes}

\sphinxstylestrong{Quality notes} \sphinxhyphen{} Insert notes related to the quality choices. Deviations from the standard tables above should be addressed and justified.


\section{Record additional information}
\label{\detokenize{RSL_data:record-additional-information}}
\sphinxstylestrong{Record created by} \sphinxhyphen{} User who created the record.

\begin{sphinxadmonition}{warning}{Warning:}
The value for this field is selected from the ‘sec\_usersusers’ table. No foreign keys are defined in the database.
\end{sphinxadmonition}

\sphinxstylestrong{Record updated by} \sphinxhyphen{} User who updated the record.

\begin{sphinxadmonition}{warning}{Warning:}
The value for this field is selected from the ‘sec\_usersusers’ table. No foreign keys are defined in the database.
\end{sphinxadmonition}

\sphinxstylestrong{Last Update} \sphinxhyphen{} Date of last update.


\chapter{Metadata}
\label{\detokenize{Metadata:metadata}}\label{\detokenize{Metadata::doc}}
\begin{sphinxadmonition}{note}{Note:}
This section has been designed by A. Rovere and D. Ryan
\end{sphinxadmonition}

This section describes the fields included in the ‘RSL indicators’,’Elevation measurement’,’Geographic positioning’ and ‘Sea level datums’ files/sheets, created respectively from the ‘rsl\_ind’,’vrtmeastech’,’hrzpostech’,’sldatum’ mySQL tables. In the interface, the fields below can be found under ‘Metadata’ page.


\section{Type of RSL indicator}
\label{\detokenize{Metadata:type-of-rsl-indicator}}
\begin{sphinxadmonition}{note}{Note:}
In the interface, any data inserted in this section is by default made public. Every user will be able to select the same sea\sphinxhyphen{}level indicator
\end{sphinxadmonition}


\subsection{Name of RSL indicator}
\label{\detokenize{Metadata:name-of-rsl-indicator}}
Short name for the RSL indicator.


\subsection{Description of RSL indicator}
\label{\detokenize{Metadata:description-of-rsl-indicator}}
Detailed descriptions of the RSL indicator.


\subsection{Description of RWL}
\label{\detokenize{Metadata:description-of-rwl}}
Description the Reference Water Level. As a reference, see Table 1 in \sphinxhref{https://www.sciencedirect.com/science/article/abs/pii/S0277379116303304}{Khan et al., 2017}. For example, use (HAT to MSL)/2 for an index point having the Highest Astronomical Tide and Mean Sea Level as, respectively, upper and lower limits of the indicative meaning.


\subsection{Description of IR}
\label{\detokenize{Metadata:description-of-ir}}
Description of the Indicative Range. As a reference, see Table 1 in \sphinxhref{https://www.sciencedirect.com/science/article/abs/pii/S0277379116303304}{Khan et al., 2017}. For example, use HAT to MSL for an index point having the Highest Astronomical Tide and Mean Sea Level as, respectively, upper and lower limits of the indicative meaning.


\subsection{Indicator reference(s)}
\label{\detokenize{Metadata:indicator-reference-s}}
Reference(s) describing the sea level indicator.

\begin{sphinxadmonition}{warning}{Warning:}
In the interface, the values for this field are selected from the ‘references’ table. The ID values are stored as comma\sphinxhyphen{}separated. No foreign keys are defined in the database.
\end{sphinxadmonition}


\section{Geographic positioning}
\label{\detokenize{Metadata:geographic-positioning}}
\begin{sphinxadmonition}{note}{Note:}
In the interface, any data inserted in this section is by default made public. Every user will be able to select the same geographic positioning technique
\end{sphinxadmonition}


\subsection{Type}
\label{\detokenize{Metadata:type}}
The type/name of the positioning technique.


\subsection{Description}
\label{\detokenize{Metadata:description}}
Description of the geographic positioning technique. Insert here details on the technique used.


\subsection{Typical accuracy}
\label{\detokenize{Metadata:typical-accuracy}}
Typical accuracy of the geographic positioning technique (free text).


\section{Sea level datum}
\label{\detokenize{Metadata:sea-level-datum}}
\begin{sphinxadmonition}{note}{Note:}
In the interface, any data inserted in this section is by default made public. Every user will be able to select the same sea level datum
\end{sphinxadmonition}


\subsection{Datum name}
\label{\detokenize{Metadata:datum-name}}
Short name for the sea level datum used.


\subsection{Datum description}
\label{\detokenize{Metadata:datum-description}}
Sea level datum description, possibly with details on how it has been calculated/derived.


\subsection{Datum uncertainty}
\label{\detokenize{Metadata:datum-uncertainty}}
Estimated or calculated datum uncertainty.


\subsection{Reference(s)}
\label{\detokenize{Metadata:reference-s}}
Reference for the sea level datum.

\begin{sphinxadmonition}{warning}{Warning:}
In the interface, the values for this field are selected from the ‘references’ table. The ID values are stored as comma\sphinxhyphen{}separated. No foreign keys are defined in the database.
\end{sphinxadmonition}


\section{Elevation measurement}
\label{\detokenize{Metadata:elevation-measurement}}
\begin{sphinxadmonition}{note}{Note:}
In the interface, any data inserted in this section is by default made public. Every user will be able to select the measurement technique
\end{sphinxadmonition}


\subsection{Measurement technique}
\label{\detokenize{Metadata:measurement-technique}}
The type/name of the measurement elevation technique.


\subsection{Description}
\label{\detokenize{Metadata:id2}}
Description of the elevation measurement technique.


\subsection{Typical accuracy}
\label{\detokenize{Metadata:id3}}
Typical accuracy of the elevation measurement technique.


\chapter{General fields for dated samples}
\label{\detokenize{Dated_samples:general-fields-for-dated-samples}}\label{\detokenize{Dated_samples::doc}}
\begin{sphinxadmonition}{note}{Note:}
This section has been designed by A. Rovere, D. Ryan, P. Chutcharavan, A. Dutton, D. Brill, M. Bartz.
\end{sphinxadmonition}

This section describes the general fields designed to collect information on U\sphinxhyphen{}Series, AAR, ESR and luminescence dated samples. These fields are included at the beginning of each file/sheet containing information on dated samples. In the interface, the fields below can be found within the ‘Analysis metadata’ and ‘Geographic metadata’ tabs of each dating technique.


\section{Analysis metadata}
\label{\detokenize{Dated_samples:analysis-metadata}}
\sphinxstylestrong{Reference(s)} \sphinxhyphen{} Reference(s) from where the U\sphinxhyphen{}Series data have been extracted.

\begin{sphinxadmonition}{warning}{Warning:}
The values for this field are selected from the ‘references’ table. The ID values are stored as comma\sphinxhyphen{}separated. No foreign keys are defined in the database.
\end{sphinxadmonition}

\sphinxstylestrong{IGSN} \sphinxhyphen{} International Geo Sample Number associated to the sample.

\sphinxstylestrong{Sample ID} \sphinxhyphen{} Sample identifier. The first two letters and numbers indicate the author and year of the study which the sample was first measured in (e.g. CH91 = Chen et al., 1991), while the following three digits indicate the sample within the study (i.e. CH91\sphinxhyphen{}001 = the first coral analyzed in Chen et al., 1991). If this is an original date, the first four characters will correspond to the author surname and the year of analysis.

\sphinxstylestrong{Analysis ID} \sphinxhyphen{} Subsample identifier. The last three digits are the subsample ID (e.g. CH91\sphinxhyphen{}001\sphinxhyphen{}002 is the second subsample/aliquot of coral CH91\sphinxhyphen{}001). If only one subsample was analysed, the last three digits must be set to “001”. If this is an original date, the first four characters will correspond to the author surname and the year of analysis.

\sphinxstylestrong{Reported ID} \sphinxhyphen{} Published sample identifier, or Laboratory ID as it appears in the original publication.

\sphinxstylestrong{Date of analysis} \sphinxhyphen{} Insert the date when the analysis was performed.

\sphinxstylestrong{Accepted?} \sphinxhyphen{} Select wether the date is accepted or rejected. In case of rejection, provide an explanation in the field below.

\sphinxstylestrong{Reason for rejection} \sphinxhyphen{} Provide an explanation for the rejection of the age.


\section{Geographic metedata}
\label{\detokenize{Dated_samples:geographic-metedata}}
\sphinxstylestrong{Latitude (decimal degrees)} \sphinxhyphen{} Insert the latitude of the sample in decimal degrees.

\sphinxstylestrong{Longitude (decimal degrees)} \sphinxhyphen{} Insert the Longitude of the sample in decimal degrees.

\sphinxstylestrong{Reported Latitude} \sphinxhyphen{} Insert the latititude as reported originally. The field accepts numbers and characters.

\sphinxstylestrong{Reported Longitude} \sphinxhyphen{} Insert the longitude as reported originally. The field accepts numbers and characters.

\sphinxstylestrong{Are Lat/Long estimated?} \sphinxhyphen{} Select ‘Yes’ if Lat and Long were not provided in the original study, but estimated a posteriori through maps or other means.

\sphinxstylestrong{Original elevation datum used} \sphinxhyphen{} Select the sea level datum to which the elevations are referred. New sea level datums can be added from the {[}Sea Level Datum{]}(Sea Level Datum.md) table.

\sphinxstylestrong{Elevation measurement method} \sphinxhyphen{} Method employed to measure elevations. New techniques can be added from the {[}Elevation Measurement{]}(Elevation measurement.md) table.

\sphinxstylestrong{Reported elevation (m)} \sphinxhyphen{} Elevation of the sample as reported in the original publication. This field accepts text values up to 50 characters. Default is N/A.

\sphinxstylestrong{Reported elevation uncertainty (m)} \sphinxhyphen{} Elevation uncertainty of the sample as reported in the original publication. This field accepts text values up to 50 characters. Default is N/A.

\sphinxstylestrong{Elevation above MSL (m)} \sphinxhyphen{} Elevation of the sample in meters above mean sea level. If not provided in the original publication, calculate it from the reported elevation and the datum used, indicating how the elevation above MSL has been derived. Default is N/A.

\sphinxstylestrong{Elevation uncertainty used (m)} \sphinxhyphen{} Final elevation uncertainty used, including measurement and datum uncertainty. Default is N/A.

\sphinxstylestrong{Elevation comments} \sphinxhyphen{} Insert here how MSL elevations have been derived from the original datum, including uncertainties.


\chapter{Samples dated with U\sphinxhyphen{}Series}
\label{\detokenize{Useries:samples-dated-with-u-series}}\label{\detokenize{Useries::doc}}
\begin{sphinxadmonition}{note}{Note:}
This section was designed by A. Dutton and P. Chutcharavan, based on a previous template by Hibbert et al. (2016, Quaternary Science Reviews)
\end{sphinxadmonition}

\begin{sphinxadmonition}{warning}{Warning:}
The U\sphinxhyphen{}Series fields displayed in the interface change according to the different possible choices (e.g., U\sphinxhyphen{}Series entered as sea level datapoint from single coral or speleothem, or U\sphinxhyphen{}Series age on corals/speleothems/algae or mollusks.
\end{sphinxadmonition}


\section{Initial choice}
\label{\detokenize{Useries:initial-choice}}
\sphinxstylestrong{Select the dated material} \sphinxhyphen{} Select which dated material will be entered. This field is automatically filled according to the selection in the main menu (corals, mollusks, algae).

\sphinxstylestrong{Details on dated material} \sphinxhyphen{} Further details on the dated material.

\sphinxstylestrong{Select whether U\sphinxhyphen{}Series data were recalculated} \sphinxhyphen{} Select if only originally reported data area available or they were recalculated.

\sphinxstylestrong{Type of spectrometry} \sphinxhyphen{} Select if alpha or mass spectrometer.

\sphinxstylestrong{Are RSL estimates for the coral available?} \sphinxhyphen{} Select if paleo RSL estimates were derived from this single coral record. This field is automatically compiled according to the choices in the interface.

\sphinxstylestrong{Do you want to insert data related to tectonics?} \sphinxhyphen{} Select if tectonic estimates are available from the single coral record.


\section{Paleo Sea Level}
\label{\detokenize{Useries:paleo-sea-level}}
\begin{sphinxadmonition}{note}{Note:}
This section is only available when RSL datapoint from single coral is selected
\end{sphinxadmonition}

\sphinxstylestrong{Original elevation datum used} \sphinxhyphen{} Select the sea level datum to which the elevations are referred

\sphinxstylestrong{Elevation measurement method} \sphinxhyphen{} How the elevation of the sample was measured

\sphinxstylestrong{Reported elevation (m)} \sphinxhyphen{} Elevation of the sample as reported in the original publication. This field accepts text values up to 50 characters. Default is N/A

\sphinxstylestrong{Reported elevation uncertainty (m)} \sphinxhyphen{} Elevation uncertainty of the sample as reported in the original publication. This field accepts text values up to 50 characters. Default is N/A

\sphinxstylestrong{Interpreted elevation relative to mean sea level (m)} \sphinxhyphen{} Elevation of the sample in meters above mean sea level. If not provided in the original publication, calculate it from the reported elevation and the datum used, indicating how the elevation above MSL has been derived.

\sphinxstylestrong{Interpreted elevation uncertainty (m)} \sphinxhyphen{} Final elevation uncertainty used, including measurement and datum uncertainty

\sphinxstylestrong{Interpreted elevation relative to MLLW/MLWS (m)} \sphinxhyphen{} Elevation relative to mean lower low water or mean low water springs (sometimes used for coral microatolls)

\sphinxstylestrong{Elevation comments} \sphinxhyphen{} Comments on elevation

\sphinxstylestrong{Published paleo water depth estimate} \sphinxhyphen{} Describe the original paleo depth interpretation

\sphinxstylestrong{Upper limit of living range} \sphinxhyphen{} Upper limit of the coral living range. This defines the minimum depth at which the coral can survive or lives in the modern reef.

\sphinxstylestrong{Lower limit of living range} \sphinxhyphen{} Lower limit of the coral living range. This defines the maximum depth at which the coral lives in the modern reef

\sphinxstylestrong{Comments on reported and estimated paleo water depth} \sphinxhyphen{} Insert here any additional comment on the paleo water depth entered above


\section{Ecological metadata}
\label{\detokenize{Useries:ecological-metadata}}
\begin{sphinxadmonition}{note}{Note:}
This section is only available under U\sphinxhyphen{}Series ages on corals or mollusks/algae
\end{sphinxadmonition}

\sphinxstylestrong{Terrace ID} \sphinxhyphen{} Identified terrace the sample came from.

\sphinxstylestrong{Facies description} \sphinxhyphen{} Sedimentary facies context and outcrop information on coral or mollusk sample.

\sphinxstylestrong{Reported as in situ?} \sphinxhyphen{} If the samples was explicitly stated as being in situ in the original publication. Note: for drill cores it must be assumed that samples are not in situ unless explicitly stated, or if referred to as “coral framework” or equivalent. If the sample is considered to be in “growth position”, then it must be also assumed to be in situ.

\sphinxstylestrong{Reported as in growth position?} \sphinxhyphen{} If the sample was explicitly stated as being in growth position in the original publication.

\sphinxstylestrong{Taxa information (as reported)} \sphinxhyphen{} Taxonomic identification of sample as reported in original manuscript.

\sphinxstylestrong{Family}      \sphinxhyphen{} Insert the family the sample belongs to.

\sphinxstylestrong{Genus} \sphinxhyphen{} Insert the genus the sample belongs to.

\sphinxstylestrong{Species}     \sphinxhyphen{} Insert the species the sample belongs to.

\sphinxstylestrong{Comments on taxa} \sphinxhyphen{} Insert any further comments on the species/genus/family reported above.


\section{Speleothem}
\label{\detokenize{Useries:speleothem}}
\begin{sphinxadmonition}{note}{Note:}
This section is only available under U\sphinxhyphen{}Series ages on speleothems
\end{sphinxadmonition}

\sphinxstylestrong{Reported in situ?} \sphinxhyphen{} Is the speleothem in original growth orientation? Note: Samples that are not in their original growth orientation (e.g. “in situ rubble”) are NOT considered in situ here.

\sphinxstylestrong{Type of deposit} \sphinxhyphen{} Type of speleothem deposit (e.g. flowstone, stalactite, stalagmite, etc.).

\sphinxstylestrong{Distance from base of deposit (m)} \sphinxhyphen{} How far along the growth axis the sample is from the base of the speleothem. If the speleothem base is at a lower elevation then the samples, use positive distance. If the speleothem base is higher in elevation, make it negative. If the deposit is at the same elevation as the base, report as positive and make a note in the “Comments (elevation)” column.

\sphinxstylestrong{Sampled material} \sphinxhyphen{} The type of material sampled (e.g. speleothem, serpulid worm tube, phreatic overgrowth on speleothem).

\sphinxstylestrong{Additional sample context} \sphinxhyphen{} Any other information that can place the sample in context to the rest of the deposit (e.g. if the sample is directly above/below a depositional hiatus)

\begin{sphinxadmonition}{note}{Note:}
The two fields below are only available when RSL datapoints from single speleothem is selected
\end{sphinxadmonition}

\sphinxstylestrong{Paleo RSL from speleothem (m)} \sphinxhyphen{} Insert the paleo RSL associated with the speleothem

\sphinxstylestrong{Paleo RSL from speleothem uncertaity (m)} \sphinxhyphen{} Insert the paleo RSL uncertainty associated with the speleothem


\section{Analytical details}
\label{\detokenize{Useries:analytical-details}}
\sphinxstylestrong{Pa/Th age?} \sphinxhyphen{} Is there is a Pa/Th age for the sample?

$^{\text{14}}$\sphinxstylestrong{C age?} \sphinxhyphen{} Is there a $^{\text{14}}$C age available for the sample?

\sphinxstylestrong{Instrument} \sphinxhyphen{} Type of mass spectrometer used for U\sphinxhyphen{}series isotopic Analysis.

\sphinxstylestrong{Decay constants} \sphinxhyphen{} Select the decay constants used as follows:
\begin{itemize}
\item {} 
D1 = $^{\text{234}}$U decay constant from Holden (1989) and $^{\text{230}}$Th decay constants of Meadows et al (1980)

\item {} 
D2 = $^{\text{234}}$U and $^{\text{230}}$Th decay constants from Cheng et al. (2000)

\item {} 
D3 = $^{\text{234}}$U and $^{\text{230}}$Th decay constants from Cheng et al. (2013)

\item {} 
No Info, therefore assumed D1

\end{itemize}

\sphinxstylestrong{Comments on decay constants} \sphinxhyphen{} Insert here any comment related to the field above.

\sphinxstylestrong{Calibration method for} $^{\text{230}}$\sphinxstylestrong{Th/} $^{\text{238}}$\sphinxstylestrong{U ratio} \sphinxhyphen{} Indicate how the $^{\text{230}}$Th/ $^{\text{238}}$U ratios of the spike were calibrated: either a secular equilibrium (SE) or gravimetric (G) standard

\sphinxstylestrong{Calibration method for} $^{\text{234}}$** U/** $^{\text{238}}$\sphinxstylestrong{U ratio} \sphinxhyphen{} Indicate how the $^{\text{234}}$U/ $^{\text{238}}$U ratios of the spike were calibrated: either a secular equilibrium (SE) or gravimetric (G) standard

\sphinxstylestrong{Comments on spike calibration} \sphinxhyphen{} Insert here comments on the fields above.\textless{}br\textgreater{}

\sphinxstylestrong{How was mineralogy determined?} \sphinxhyphen{} List any methods used to determine the composition of the sampled material (e.g. XRD, ICP\sphinxhyphen{}MS, thin section)

\sphinxstylestrong{Other screening techniques applied} \sphinxhyphen{} Other approaches besides XRD that were used to assess sample preservation

\sphinxstylestrong{Published \% calcite} \sphinxstyleemphasis{(Only for corals and mollusks)} \sphinxhyphen{} Reported \% calcite content of the sample determined via X\sphinxhyphen{}ray diffraction (XRD).

\sphinxstylestrong{Interpreted \% calcite} \sphinxstyleemphasis{(Only for corals and mollusks)} \sphinxhyphen{} Interpretation of calcite content based on what was reported for ‘\% calcite’.

\sphinxstylestrong{Other screening techniques applied} \sphinxhyphen{} Other approaches besides XRD that were used to assess sample preservation (incorporates older information from “determined by” column in Hibbert et al., 2016)


\section{Analytical data}
\label{\detokenize{Useries:analytical-data}}
\begin{sphinxadmonition}{note}{Note:}
Insert the analytical data if available.
\end{sphinxadmonition}

\sphinxstylestrong{{[}} $^{\text{230}}$\sphinxstylestrong{Th/} $^{\text{232}}$\sphinxstylestrong{Th{]}} $_{\text{ACT}}$\sphinxstylestrong{backcalculated?}

\sphinxstylestrong{{[}} $^{\text{232}}$\sphinxstylestrong{Th/} $^{\text{238}}$\sphinxstylestrong{U{]}} $_{\text{ACT}}$\sphinxstylestrong{backcalculated?}

\sphinxstylestrong{Atomic ratio (} $^{\text{232}}$\sphinxstylestrong{Th/} $^{\text{238}}$\sphinxstylestrong{U)x10} $^{\text{5}}$

\sphinxstylestrong{Initial} $^{\text{230}}$\sphinxstylestrong{Th/} $^{\text{232}}$\sphinxstylestrong{Th}

\sphinxstylestrong{{[}} $^{\text{232}}$\sphinxstylestrong{Th{]} (ppb)}

\sphinxstylestrong{{[}} $^{\text{232}}$\sphinxstylestrong{Th{]} (ppb) uncertainty (\(\pm\)2\sphinxhyphen{}sigma)}

\sphinxstylestrong{{[}} $^{\text{230}}$\sphinxstylestrong{Th/} $^{\text{232}}$\sphinxstylestrong{Th{]}} $_{\text{ACT}}$

\sphinxstylestrong{{[}} $^{\text{230}}$\sphinxstylestrong{Th/} $^{\text{232}}$\sphinxstylestrong{Th{]}} $_{\text{ACT}}$\sphinxstylestrong{uncertainty (\(\pm\)2\sphinxhyphen{}sigma)}

\sphinxstylestrong{{[}} $^{\text{232}}$\sphinxstylestrong{Th/} $^{\text{238}}$\sphinxstylestrong{U{]}} $_{\text{ACT}}$

\sphinxstylestrong{{[}} $^{\text{232}}$\sphinxstylestrong{Th/} $^{\text{238}}$\sphinxstylestrong{U{]}} $_{\text{ACT}}$\sphinxstylestrong{uncertainty (\(\pm\)2\sphinxhyphen{}sigma)}

\sphinxstylestrong{{[}} $^{\text{238}}$\sphinxstylestrong{U{]} (ppm)}

\sphinxstylestrong{{[}} $^{\text{238}}$\sphinxstylestrong{U{]} (ppm) uncertainty (\(\pm\)2\sphinxhyphen{}sigma)}


\section{Originally reported}
\label{\detokenize{Useries:originally-reported}}
\begin{sphinxadmonition}{note}{Note:}
Insert the following values, as reported in the original study
\end{sphinxadmonition}

\sphinxstylestrong{{[}} $^{\text{230}}$\sphinxstylestrong{Th/} $^{\text{238}}$\sphinxstylestrong{U{]}} $_{\text{ACT}}$\sphinxstylestrong{backcalculated?}

\sphinxstylestrong{{[}} $^{\text{234}}$\sphinxstylestrong{Th/} $^{\text{238}}$\sphinxstylestrong{U{]}} $_{\text{ACT}}$\sphinxstylestrong{backcalculated?}

\sphinxstylestrong{{[}} $^{\text{230}}$\sphinxstylestrong{Th/} $^{\text{234}}$\sphinxstylestrong{U{]}} $_{\text{ACT}}$

\sphinxstylestrong{{[}} $^{\text{230}}$\sphinxstylestrong{Th/} $^{\text{234}}$\sphinxstylestrong{U{]}} $_{\text{ACT}}$\sphinxstylestrong{uncertainty (\(\pm\)2\sphinxhyphen{}sigma)}

\sphinxstylestrong{{[}} $^{\text{230}}$\sphinxstylestrong{Th/} $^{\text{238}}$\sphinxstylestrong{U{]}} $_{\text{ACT}}$

\sphinxstylestrong{{[}} $^{\text{230}}$\sphinxstylestrong{Th/} $^{\text{238}}$\sphinxstylestrong{U{]}} $_{\text{ACT}}$\sphinxstylestrong{uncertainty (\(\pm\)2\sphinxhyphen{}sigma)}

\sphinxstylestrong{{[}} $^{\text{234}}$\sphinxstylestrong{Th/} $^{\text{238}}$\sphinxstylestrong{U{]}} $_{\text{ACT}}$

\sphinxstylestrong{{[}} $^{\text{234}}$\sphinxstylestrong{Th/} $^{\text{238}}$\sphinxstylestrong{U{]}} $_{\text{ACT}}$\sphinxstylestrong{uncertainty (\(\pm\)2\sphinxhyphen{}sigma)}

\sphinxstylestrong{Reported age (ka)} \sphinxstyleemphasis{(Only for corals and speleothems)}

\sphinxstylestrong{Reported age uncertainty (ka, \(\pm\)2\sphinxhyphen{}sigma)} \sphinxstyleemphasis{(Only for corals and speleothems)}

\sphinxstylestrong{Corrected reported age (ka)} \sphinxhyphen{} \sphinxstyleemphasis{(Only for speleothems)}

\sphinxstylestrong{Corrected reported age uncertainty (ka)} \sphinxhyphen{} \sphinxstyleemphasis{(Only for speleothems)}

\sphinxstylestrong{Initial} $^{\text{234}}$\sphinxstylestrong{U/} $^{\text{238}}$\sphinxstylestrong{U}

\sphinxstylestrong{Initial} $^{\text{234}}$\sphinxstylestrong{U/} $^{\text{238}}$\sphinxstylestrong{U uncertainty (\(\pm\)2\sphinxhyphen{}sigma)}

\sphinxstylestrong{Measured} $^{\text{234}}$\sphinxstylestrong{U/} $^{\text{238}}$\sphinxstylestrong{U}

\sphinxstylestrong{Measured} $^{\text{234}}$\sphinxstylestrong{U/} $^{\text{238}}$\sphinxstylestrong{U uncertainty (\(\pm\)2\sphinxhyphen{}sigma)}

\sphinxstylestrong{Reported delta} $^{\text{234}}$\sphinxstylestrong{U initial (‰)}

\sphinxstylestrong{Reported delta} $^{\text{234}}$\sphinxstylestrong{U (per mille) uncertainty (\(\pm\)2\sphinxhyphen{}sigma)}

\sphinxstylestrong{Measured delta} $^{\text{234}}$\sphinxstylestrong{U initial (per mille)}

\sphinxstylestrong{Measured delta} $^{\text{234}}$\sphinxstylestrong{U (per mille) uncertainty (\(\pm\)2\sphinxhyphen{}sigma)}

\sphinxstylestrong{Comments on reported age}


\section{HU\sphinxhyphen{}1 spike correction}
\label{\detokenize{Useries:hu-1-spike-correction}}
\sphinxstylestrong{Reference material name for} $^{\text{230}}$\sphinxstylestrong{Th/} $^{\text{238}}$\sphinxstylestrong{U} \sphinxhyphen{} Name of standard if SE standard used to calibrate activity ratio.

\sphinxstylestrong{Reference material name for} $^{\text{234}}$\sphinxstylestrong{U/} $^{\text{238}}$\sphinxstylestrong{U} \sphinxhyphen{} Name of standard if SE standard used to calibrate activity ratio.

\sphinxstylestrong{Correction factor for} $^{\text{230}}$\sphinxstylestrong{Th/} $^{\text{238}}$\sphinxstylestrong{U} \sphinxhyphen{} HU\sphinxhyphen{}1 correction factor.

\sphinxstylestrong{Correction factor for} $^{\text{230}}$\sphinxstylestrong{Th/} $^{\text{238}}$\sphinxstylestrong{U uncertainty (\(\pm\)2\sphinxhyphen{}sigma)} \sphinxhyphen{} HU\sphinxhyphen{}1 correction factor uncertainty.

\sphinxstylestrong{Correction factor for} $^{\text{234}}$\sphinxstylestrong{U/} $^{\text{238}}$\sphinxstylestrong{U} \sphinxhyphen{} HU\sphinxhyphen{}1 correction factor.

\sphinxstylestrong{Correction factor for} $^{\text{234}}$\sphinxstylestrong{U/} $^{\text{238}}$\sphinxstylestrong{U uncertainty (\(\pm\)2\sphinxhyphen{}sigma)} \sphinxhyphen{} HU\sphinxhyphen{}1 correction factor uncertainty.


\section{Recalculated}
\label{\detokenize{Useries:recalculated}}
\begin{sphinxadmonition}{note}{Note:}
Insert the following values, if recalculated from the original study
\end{sphinxadmonition}

\sphinxstylestrong{{[}} $^{\text{230}}$\sphinxstylestrong{Th/} $^{\text{238}}$\sphinxstylestrong{U{]}} $_{\text{ACT}}$

\sphinxstylestrong{{[}} $^{\text{230}}$\sphinxstylestrong{Th/} $^{\text{238}}$\sphinxstylestrong{U{]}} $_{\text{ACT}}$\sphinxstylestrong{uncertainty (\(\pm\)2\sphinxhyphen{}sigma)}

\sphinxstylestrong{{[}} $^{\text{234}}$\sphinxstylestrong{Th/} $^{\text{238}}$\sphinxstylestrong{U{]}} $_{\text{ACT}}$

\sphinxstylestrong{{[}} $^{\text{234}}$\sphinxstylestrong{Th/} $^{\text{238}}$\sphinxstylestrong{U{]}} $_{\text{ACT}}$\sphinxstylestrong{uncertainty (\(\pm\)2\sphinxhyphen{}sigma)}

\sphinxstylestrong{Recalculated Conventional Age (ka)} \sphinxstyleemphasis{(Only for corals and speleothems)}

\sphinxstylestrong{Recalculated Conventional Age uncert. (\(\pm\)2\sphinxhyphen{}sigma)} \sphinxstyleemphasis{(Only for corals and speleothems)}

\sphinxstylestrong{Recalculated delta 234Ui (per mille)}

\sphinxstylestrong{Recalculated delta 234Ui uncertainty (\(\pm\)2\sphinxhyphen{}sigma)}

\sphinxstylestrong{Recalculated Conventional Age uncert. w/ decay constant uncertainties (\(\pm\)2\sphinxhyphen{}sigma)}

\sphinxstylestrong{Recalculated delta 234Ui uncert. (\(\pm\)2\sphinxhyphen{}sigma) w/ decay constant uncertainties}

\sphinxstylestrong{Comments (age and delta} $^{\text{234}}$\sphinxstylestrong{Ui)}


\section{Age constraints}
\label{\detokenize{Useries:age-constraints}}
\begin{sphinxadmonition}{warning}{Warning:}
The fields below can be filled only for U\sphinxhyphen{}series ages on algae or mollusks
\end{sphinxadmonition}

\sphinxstylestrong{Age is Older/Equal/Younger than} \sphinxhyphen{} Select one option among those listed.

\sphinxstylestrong{Marine Isotopic Stage} \sphinxhyphen{} Select a MIS from the list. Add new MIS definitions if necessary. If the MIS is already present in the list but a different definition is needed, use the text box below (Comments/details on MIS designation).

\sphinxstylestrong{Age determination} \sphinxhyphen{} Provide comments or details on the designation above.


\chapter{Samples dated with AAR}
\label{\detokenize{AAR:samples-dated-with-aar}}\label{\detokenize{AAR::doc}}
\begin{sphinxadmonition}{note}{Note:}
This section was designed by D. Ryan
\end{sphinxadmonition}


\section{Age}
\label{\detokenize{AAR:age}}
\sphinxstylestrong{Age is Older/Equal/Younger than} \sphinxhyphen{} Select one option among those listed.

\sphinxstylestrong{Marine Isotopic Stage} \sphinxhyphen{} Select a MIS from the list. Add new MIS definitions if necessary. If the MIS is already present in the list but a different definition is needed, use the text box below (Comments/details on MIS designation).

\sphinxstylestrong{Comments/details on MIS designation} \sphinxhyphen{} Provide comments or details on the designation above.

\sphinxstylestrong{Reported age (ka)} \sphinxhyphen{} Reported age.

\sphinxstylestrong{Reported age uncertainty (\(\pm\)2\sphinxhyphen{}sigma) (ka)} \sphinxhyphen{} Reported age uncertainty. If the uncertainty level is not reported, note it in the textbox above. If reported to \(\pm\)1\sphinxhyphen{}sigma level, transform to \(\pm\)2\sphinxhyphen{}sigma.


\section{Detailed sample and analysis information}
\label{\detokenize{AAR:detailed-sample-and-analysis-information}}
\sphinxstylestrong{Sample Type} \sphinxhyphen{} Indicate whether marine/estuarine/freshwater mollusc or gastropod shell, foraminifer, or whole\sphinxhyphen{}rock sample.

\sphinxstylestrong{Taxonomy} \sphinxhyphen{} If known or applicable, provide at least the Genus. Description to the species level is beneficial.

\sphinxstylestrong{Current Annual Mean Temperature (deg C)} \sphinxhyphen{} The mean annual temperature for the field location as determined from the national government bureau of meteorology or equivalent.

\sphinxstylestrong{Collection Context} \sphinxhyphen{} Brief description of taphonomic condition and depositional setting/sedimentary environment/facies; e.g. mollusc shell in live position, robust condition with little or no evidence of abrasion, in back\sphinxhyphen{}barrier lagoon sediments

\sphinxstylestrong{Depth of burial (m)} \sphinxhyphen{} Depth of burial below modern surface. If present, include proximity to over\sphinxhyphen{} or underlying diastems and/or unconformities. If unknown digit “unknown”.

\sphinxstylestrong{Preparation procedure} \sphinxhyphen{} Reference to sample preparation procedure.

\sphinxstylestrong{Method of chromatography} \sphinxhyphen{} E.g. gas chromatography (GC), reversed\sphinxhyphen{}phase high\sphinxhyphen{}performance liquid chromatography (RP\sphinxhyphen{}HPLC), or ion\sphinxhyphen{}exchange chromatography high performance liquid chromatography (IEC HPLC). If unknown, digit “unknown”

\sphinxstylestrong{Stationary phase of the column} \sphinxhyphen{} Describe the stationary phase of the column.

\sphinxstylestrong{Notes and comments on sample quality} \sphinxhyphen{} Any additional notes concerning sample that may have impact of sample results.

\sphinxstylestrong{Sample position} \sphinxhyphen{} Indicate what portion of the shell was sampled, e.g. lip, hinge, etc.

\sphinxstylestrong{Whole\sphinxhyphen{}Rock?} \sphinxhyphen{} Indicate if the record refers to a whole\sphinxhyphen{}rock analysis.

\begin{sphinxadmonition}{warning}{Warning:}
The following fields appear if “Yes” is selected above:

\sphinxstylestrong{Grain Size (micro\sphinxhyphen{}m)}

\sphinxstylestrong{Carbonate (\%)}

\sphinxstylestrong{Fraction (micro\sphinxhyphen{}m)}
\end{sphinxadmonition}

\sphinxstylestrong{Number of replicates (samples)}

\sphinxstylestrong{Number of sample replicates (injections)}

\sphinxstylestrong{Ratio type} \sphinxhyphen{} Was peak height or peak area reported?

\sphinxstylestrong{Pyrolysis experiment available} \sphinxhyphen{} Indicate if the results of a pyrolysis (isothermal heating) experiment are used in discussion of field sample results and reference publication


\section{Reported Amino Acid D/L Values (THAA)}
\label{\detokenize{AAR:reported-amino-acid-d-l-values-thaa}}
\sphinxstylestrong{Level of uncertainty} \sphinxhyphen{} Result uncertainty, e.g. 1\sphinxhyphen{}sigma or 2\sphinxhyphen{}sigma or N/A.

\sphinxstyleemphasis{Amino Acid D/L values:}

\sphinxstylestrong{Aspartic} / \sphinxstylestrong{Aspartic uncertainty}

\sphinxstylestrong{Glutamic} / \sphinxstylestrong{Glutamic uncertainty}

\sphinxstylestrong{Serine} / \sphinxstylestrong{Serine uncertainty}

\sphinxstylestrong{Alanine} / \sphinxstylestrong{Alanine uncertainty}

\sphinxstylestrong{Valine} / \sphinxstylestrong{Valine uncertainty}

\sphinxstylestrong{A/I} / \sphinxstylestrong{A/I uncertainty}

\sphinxstylestrong{Leucine} / \sphinxstylestrong{Leucine uncertainty}

\sphinxstylestrong{Notes/Comments} \sphinxhyphen{} Any additional information that may be worthy of mention

\begin{sphinxadmonition}{note}{Note:}
The following selections define which fields must be filled next:

\sphinxstylestrong{Independent age constraints available?}

\sphinxstylestrong{Calibration data available?}

\sphinxstylestrong{Are free AAR ratios available?}
\end{sphinxadmonition}


\section{Reported Amino Acid D/L Values (FAA)}
\label{\detokenize{AAR:reported-amino-acid-d-l-values-faa}}
\begin{sphinxadmonition}{warning}{Warning:}
This tab is activated only if ‘Yes’ is selected in the ‘Are free AAR ratios available?’ field
\end{sphinxadmonition}

\sphinxstylestrong{Level of uncertainty} \sphinxhyphen{} Insert uncertainty, e.g. 1\sphinxhyphen{}sigma or 2\sphinxhyphen{}sigma or N/A.

\sphinxstyleemphasis{Amino Acid D/L values:}

\sphinxstylestrong{Aspartic} / \sphinxstylestrong{Aspartic uncertainty}

\sphinxstylestrong{Glutamic} / \sphinxstylestrong{Glutamic uncertainty}

\sphinxstylestrong{Serine} / \sphinxstylestrong{Serine uncertainty}

\sphinxstylestrong{Alanine} / \sphinxstylestrong{Alanine uncertainty}

\sphinxstylestrong{Valine} / \sphinxstylestrong{Valine uncertainty}

\sphinxstylestrong{A/I} / \sphinxstylestrong{A/I uncertainty}

\sphinxstylestrong{Leucine} / \sphinxstylestrong{Leucine uncertainty}

\sphinxstylestrong{Notes/Comments} \sphinxhyphen{} Any additional information that may be worthy of mention


\section{Calibration}
\label{\detokenize{AAR:calibration}}
\begin{sphinxadmonition}{warning}{Warning:}
This field is activated only if ‘Yes’ is selected in the ‘Calibration data available?’ field
\end{sphinxadmonition}

\sphinxstylestrong{Calibration method} \sphinxhyphen{} Method for calibrating age. The most common approach is to subdivide a single sample, and analyze one subsample with AAR and the second with the complementary method.

\sphinxstylestrong{Amino Acid age equation} \sphinxhyphen{} If the D/L values are calibrated to provide an age, what equation is used.

\sphinxstylestrong{Calibration laboratory} \sphinxhyphen{} Laboratory where the age calibration was done.

\sphinxstylestrong{Calibration laboratory ID} \sphinxhyphen{} Original ID of the age calibration.

\sphinxstylestrong{Calibration reference} \sphinxhyphen{} Select a reference for the age calibration.

\sphinxstylestrong{Calibration D/L value} / \sphinxstylestrong{Calibration D/L value uncertainty (\(\pm\)1\sphinxhyphen{}sigma)}

\sphinxstylestrong{Calibration age (ka)} / \sphinxstylestrong{Calibration age (ka) uncertainty (\(\pm\)1\sphinxhyphen{}sigma)}


\section{Independent Age}
\label{\detokenize{AAR:independent-age}}
\begin{sphinxadmonition}{warning}{Warning:}
This field is activated only if ‘Yes’ is selected in the ‘Independent age constraints available?’ field
\end{sphinxadmonition}

\sphinxstylestrong{Independent age choice} \sphinxhyphen{} Select the independent age constraint available. Option will appear according to the selected value in this field.

\sphinxstyleemphasis{The following selections appear depending on the choice made above:}

\sphinxstylestrong{Independent age method} \sphinxhyphen{} Describe briefly the independent age determination methodology.

\sphinxstylestrong{Independent age laboratory} \sphinxhyphen{} Insert the name of the laboratory where the independent age has been obtained.

\sphinxstylestrong{Independent age laboratory ID} \sphinxhyphen{} Laboratory ID for the independent age.

\sphinxstylestrong{Independent age result} \sphinxhyphen{} Indicate here the independent age and associated uncertainties.

\sphinxstylestrong{Independent age reference} \sphinxhyphen{} Select a reference for the independent age.

\sphinxstylestrong{Useries constraint} \sphinxhyphen{} If a U\sphinxhyphen{}Series independent constraint is available, select the ID here.

\sphinxstylestrong{ESR constraint} \sphinxhyphen{} If an ESR independent constraint is available, select the ID here.

\sphinxstylestrong{Luminescence constraint} \sphinxhyphen{} If a luminescence independent constraint is available, select the ID here.

\sphinxstylestrong{Stratigraphic constraint} \sphinxhyphen{} If a stratigraphic independent constraint is available, select the ID here.


\chapter{Samples dated with luminescence}
\label{\detokenize{Luminescence:samples-dated-with-luminescence}}\label{\detokenize{Luminescence::doc}}
\begin{sphinxadmonition}{note}{Note:}
This section was designed by D. Brill, N. Jankowski, D. Mueller.
\end{sphinxadmonition}


\section{Age}
\label{\detokenize{Luminescence:age}}
\sphinxstylestrong{Age is Older/Equal/Younger than} \sphinxhyphen{} Select one option among those listed.

\sphinxstylestrong{Marine Isotopic Stage} \sphinxhyphen{} Select a MIS from the list. Add new MIS definitions if necessary. If the MIS is already present in the list but a different definition is needed, use the text box below (Comments/details on MIS designation).

\sphinxstylestrong{Reported age (ka)} \sphinxhyphen{} Reported age.

\sphinxstylestrong{Reported age uncertainty (\(\pm\)2\sphinxhyphen{}sigma) (ka)} \sphinxhyphen{} Reported age uncertainty. If the uncertainty level is not reported, note it in the textbox above. If reported to \(\pm\)1\sphinxhyphen{}sigma level, transform to \(\pm\)2\sphinxhyphen{}sigma.

\sphinxstylestrong{Comments/details on MIS designation} \sphinxhyphen{} Provide comments or details on the designation above.


\section{Dose estimation}
\label{\detokenize{Luminescence:dose-estimation}}
\sphinxstylestrong{Mineral type} \sphinxhyphen{} Designate from which mineral luminosity was measured.

\sphinxstylestrong{Grain size} \sphinxhyphen{} Select a grain size.

\sphinxstylestrong{Aliquot size (mg)} \sphinxhyphen{} Describe aliquot size or indicate if single grains were measured; e.g. 5mm aliquot, 1mm aliquot, single grain.

\sphinxstylestrong{Equivalent Dose measurement protocol} \sphinxhyphen{} Provide the name of the procedure used; e.g. single aliquot regenerative dose (SAR).

\sphinxstylestrong{Treatment during measurement} \sphinxhyphen{} Treatment during measurement (e.g. preheat, cutheat and stimulation temperatures)

\sphinxstylestrong{Statistical model for burial dose} \sphinxhyphen{} Example: Central age model or Minimum age model.

\sphinxstylestrong{Machine type} \sphinxhyphen{} Provide machine type; e.g. Ris¸ TL/OSL DA\sphinxhyphen{}12.

\sphinxstylestrong{Detection unit} \sphinxhyphen{} Provide type of photo multiplier tube (e.g. EMI 9235AQ) or camera type.

\sphinxstylestrong{Stimulation unit}

\sphinxstylestrong{Optical Filters used for Detection} \sphinxhyphen{} Describe optical filters used for detection.

\sphinxstylestrong{Number of aliquots/ single grains run} \sphinxhyphen{} Provide the total number of grains or aliquots measured.

\sphinxstylestrong{Number of aliquots/ single grains used in sample} \sphinxhyphen{} Provide the number of grains used after the rejection criteria have been applied.

\sphinxstylestrong{Uncertainty level} \sphinxhyphen{} Select the uncertainty level for equivalent dose  and overdispersion values.

\sphinxstylestrong{Equivalent dose (Gy)} / \sphinxstylestrong{Equivalent doese uncertainty (Gy)}

\sphinxstylestrong{Overdispersion value (\%)} / \sphinxstylestrong{Overdispersion value uncertainty}


\section{Dose rates}
\label{\detokenize{Luminescence:dose-rates}}
\sphinxstylestrong{Approach of dose rate dermination}

\sphinxstylestrong{Depth of burial (m)} \sphinxhyphen{} Depth of burial below modern surface. If present, include proximity to over\sphinxhyphen{} or underlying diastems and/or unconformities.

\sphinxstylestrong{Uncertainty levels} \sphinxhyphen{} Indicate the uncertainty level for the values in this tab.

\sphinxstyleemphasis{Dose rate values:}

\sphinxstylestrong{Uranium content (ppm)} / \sphinxstylestrong{Uranium content uncertainty (ppm)}

\sphinxstylestrong{Thorium content (ppm)} / \sphinxstylestrong{Thorium content uncertainty (ppm)}

\sphinxstylestrong{Potassium content (\%)} / \sphinxstylestrong{Potassium content uncertainty (\%)}

\sphinxstylestrong{Rubidium content (ppm)} / \sphinxstylestrong{Rubidium content uncertainty (ppm)}

\sphinxstylestrong{Alpha dose rate (Gy/ka)} / \sphinxstylestrong{Alpha dose rate uncertainty (Gy/ka)}

\sphinxstylestrong{Gamma dose rate (Gy/ka)} / \sphinxstylestrong{Gamma dose rate uncertainty (Gy/ka)}

\sphinxstylestrong{Beta dose rate (Gy/ka)} / \sphinxstylestrong{Beta dose rate uncertainty (Gy/ka)}

\sphinxstylestrong{Cosmic dose rate (Gy/ka)} / \sphinxstylestrong{Cosmic dose rate uncertainty (Gy/ka)}

\sphinxstylestrong{Water content determination method} \sphinxhyphen{} Provide method name and reference.

\sphinxstylestrong{Field water content (\% of dry mass)} \sphinxhyphen{} Provide as a percentage and method of determination.

\sphinxstylestrong{Water Content used in Final Age Calculation (\% of dry mass)} / \sphinxstylestrong{Water Content used in Final Age Calculation uncertainty (\% of dry mass)}

\sphinxstylestrong{Alpha attenuation factor} \sphinxhyphen{} Provide reference to attenuation factors method.

\sphinxstylestrong{Dose Rate conversion factors} \sphinxhyphen{} Provide reference to dose rate conversion factors.

\sphinxstylestrong{Total dose rate Gy/Ka} / \sphinxstylestrong{Total dose rate uncertainty (Gy/ka)}


\section{Other details}
\label{\detokenize{Luminescence:other-details}}
\sphinxstylestrong{Comments and notes} \sphinxhyphen{} Insert here any comments or notes on the ESR analysis.


\chapter{Samples dated with ESR}
\label{\detokenize{ESR:samples-dated-with-esr}}\label{\detokenize{ESR::doc}}
\begin{sphinxadmonition}{note}{Note:}
This section was designed by M. Bartz.
\end{sphinxadmonition}


\section{Age}
\label{\detokenize{ESR:age}}
\sphinxstylestrong{Age is Older/Equal/Younger than} \sphinxhyphen{} Select one option among those listed.

\sphinxstylestrong{Marine Isotopic Stage} \sphinxhyphen{} Select a MIS from the list. Add new MIS definitions if necessary. If the MIS is already present in the list but a different definition is needed, use the text box below (Comments/details on MIS designation).

\sphinxstylestrong{Reported age (ka)} \sphinxhyphen{} Reported age.

\sphinxstylestrong{Reported age uncertainty (\(\pm\)2\sphinxhyphen{}sigma) (ka)} \sphinxhyphen{} Reported age uncertainty. If the uncertainty level is not reported, note it in the textbox below. If reported to \(\pm\)1\sphinxhyphen{}sigma level, transform to \(\pm\)2\sphinxhyphen{}sigma.

\sphinxstylestrong{Comments/details on MIS designation} \sphinxhyphen{} Provide comments or details on the designation above.


\section{Dose estimation}
\label{\detokenize{ESR:dose-estimation}}
\sphinxstylestrong{Mineral type} \sphinxhyphen{} Designate from which mineral the ESR intensity was measured.

\sphinxstylestrong{Grain size (µm)} \sphinxhyphen{} Designate which grain size was measured, e.g. 100\sphinxhyphen{}200 µm.

\sphinxstylestrong{Aliquot size (mg)} \sphinxhyphen{} Designate which aliquot size was used: multi\sphinxhyphen{}grain or single\sphinxhyphen{}grain aliquots. If multi\sphinxhyphen{}grain aliquots are used, provide the aliquot weight (e.g. 100 mg).

\sphinxstylestrong{Measurement protocol} \sphinxhyphen{} Provide information on the used measurement protocol, e.g. single aliquot regenerative dose (SAR), single aliquot additive dose (SAAD), multiple aliquot additive dose (MAAD).

\sphinxstylestrong{ESR spectrometre type} \sphinxhyphen{} Designate the ESR spectrometre type (e.g. Bruker X\sphinxhyphen{}band ELEXSYS E500). Provide information on the variable temperature unit (VTU) and resonator.

\sphinxstylestrong{Measurement temperature (K)} \sphinxhyphen{} Designate at which temperature the ESR intensity was measured: Room temperature or low temperature.

\sphinxstylestrong{Acquistion parameters} \sphinxhyphen{} Designate the experimental conditions: Frequency, microwave power, points resolution, sweep width, modulation frequency, modulation amplitude, conversion time, time constant.

\sphinxstylestrong{ESR signal} \sphinxhyphen{} Provide information on the g\sphinxhyphen{}factor and/or paramagnetic center.

\sphinxstylestrong{Number of scans} \sphinxhyphen{} Designate which centre was targeted, e.g. Al centre in quartz.

\sphinxstylestrong{Number of repeated measurements} \sphinxhyphen{} Designate which centre was targeted, e.g. Al centre in quartz.

\sphinxstylestrong{Number of rotations in the cavity} \sphinxhyphen{} Designate how many rotations in the cavity were used.

\sphinxstylestrong{Evaluation of the ESR signal intensity} \sphinxhyphen{} Designate how ESR intensities were evaluated, e.g. peak\sphinxhyphen{}to\sphinxhyphen{}peak measurements, peak\sphinxhyphen{}to\sphinxhyphen{}baseline, deconvolution (provide information on the used g\sphinxhyphen{}factors).

\sphinxstylestrong{Correction of the ESR intensities} \sphinxhyphen{} Designate whether ESR intensities were corrected, e.g. by sample weight, temperature correction factors, receiver gain, number of scans.

\sphinxstylestrong{Bleaching coefficient (\%)} \sphinxhyphen{} Provide information on beaching conditions, e.g. for the Al centre in quartz.

\begin{sphinxadmonition}{note}{Note:}
This field appears only if “Quartz” is selected as material
\end{sphinxadmonition}

\sphinxstylestrong{Fitting program and error calculation} \sphinxhyphen{} Designate which software was used to fit the dose response curve.

\sphinxstylestrong{Fitting function} \sphinxhyphen{} Designate the fitting function that was used for dose calculation, e.g. single saturating exponential (SSE) function.

\sphinxstylestrong{Data weighting used for fitting} \sphinxhyphen{} Provide information on data weighting, e.g. equal weights or weighting by the inverse of the squared ESR intensity (1/I\textless{}sup\textgreater{}2\textless{}/sup\textgreater{}).

\sphinxstylestrong{Equivalent dose estimate (Gy)} / \sphinxstylestrong{Equivalent dose estimate undertainty (\(\pm\)1\sphinxhyphen{}sigma) (Gy)}


\section{Dose rates}
\label{\detokenize{ESR:dose-rates}}
\sphinxstylestrong{Depth (m)} \sphinxhyphen{} Depth below surface.

\sphinxstylestrong{Technique} \sphinxhyphen{} Provide information on techniques that determine radioelement concentrations or total alpha, beta or gamma dose rate values.

\sphinxstylestrong{Dose rate uncertainty type} \sphinxhyphen{} Insert uncertainty, e.g. 1\sphinxhyphen{}sigma or 2\sphinxhyphen{}sigma or N/A.

\sphinxstyleemphasis{Dose rate values:}

\sphinxstylestrong{external U (ppm)} / \sphinxstylestrong{external U error (ppm)}

\sphinxstylestrong{external Th (ppm)} / \sphinxstylestrong{external Th error (ppm)}

\sphinxstylestrong{external K (\%)} / \sphinxstylestrong{external K error (\%)}

\sphinxstylestrong{external Rb (ppm)} / \sphinxstylestrong{external Rb error (ppm)}

\sphinxstylestrong{External dose rate (Gy/ka)} / \sphinxstylestrong{External dose rate uncertainty (Gy/ka)}

\sphinxstylestrong{internal U (ppm)} / \sphinxstylestrong{internal U error (ppm)}

\sphinxstylestrong{internal Th (ppm)} / \sphinxstylestrong{internal Th error (ppm)}

\sphinxstylestrong{internal K (\%)} / \sphinxstylestrong{internal K error (\%)}

\sphinxstylestrong{internal Rb (ppm)} / \sphinxstylestrong{internal Rb error (ppm)}

\sphinxstylestrong{Internal dose rate (Gy/ka)} / \sphinxstylestrong{Internal dose rate (Gy/ka)}

\sphinxstylestrong{Alpha dose rate (Gy/ka)} / \sphinxstylestrong{Alpha dose rate error (Gy/ka)}

\sphinxstylestrong{Gamma dose rate (Gy/ka)} / \sphinxstylestrong{Gamma dose rate error (Gy/ka)}

\sphinxstylestrong{Beta dose rate (Gy/ka)} / \sphinxstylestrong{Beta dose rate error (Gy/ka)}

\sphinxstylestrong{Cosmic dose rate (Gy/ka)} / \sphinxstylestrong{Cosmic dose rate error (Gy/ka)}

\sphinxstylestrong{Water content determination method}

\sphinxstylestrong{Field water content (\% of dry mass)}

\sphinxstylestrong{Water content used (\%)}

\sphinxstylestrong{Water content used error (\%)}

\sphinxstylestrong{Attenuation factors} \sphinxhyphen{} Provide full reference for the attenuation factor(s) used.

\sphinxstylestrong{Conversion factors} \sphinxhyphen{} Provide full reference for the conversion factor(s) used.

\sphinxstylestrong{Alpha efficiency factor} \sphinxhyphen{} Designate which alpha efficiency factor was used (provide reference).

\sphinxstylestrong{Etching method} \sphinxhyphen{} Designate how the samples were etched (by e.g. HF).

\sphinxstylestrong{Etch depth (µm)} \sphinxhyphen{} Provide the depth of the etched surface.

\sphinxstylestrong{Etch depth error (µm)}

\sphinxstylestrong{Total dose rate (Gy/ka)} / \sphinxstylestrong{Total dose rate uncertainty (Gy/ka)}


\section{Other details}
\label{\detokenize{ESR:other-details}}
\sphinxstylestrong{Comments and notes} \sphinxhyphen{} Insert here any comments or notes on the ESR analysis.


\chapter{Chronostratigraphic constraints}
\label{\detokenize{Chronostratigraphy:chronostratigraphic-constraints}}\label{\detokenize{Chronostratigraphy::doc}}
\begin{sphinxadmonition}{note}{Note:}
This section was designed by A. Rovere and K. Cohen.
\end{sphinxadmonition}


\section{Description}
\label{\detokenize{Chronostratigraphy:description}}
\sphinxstylestrong{Chronostratigraphy ID} \sphinxhyphen{} Name of the stratigraphic unit or facies. This name will be selectable in other forms.

\sphinxstylestrong{Is this an official stratigraphic designation?} \sphinxhyphen{} Indicate if this is an official stratigraphic designation, e.g. adopted by a national geological survey.

\sphinxstylestrong{Reference(s)}\sphinxhyphen{} Select the reference(s) from where stratigraphic information has been extracted.

\sphinxstylestrong{Description for chronostratigraphic constraint} \sphinxhyphen{} Describe the lithofacies, including any relevant ecological or sedimentary properties that are used to define its age.

\sphinxstylestrong{Unit thickness (m)} \sphinxhyphen{} Thickness of the unit in meters. Free text is also allowed.


\section{Age}
\label{\detokenize{Chronostratigraphy:age}}
\sphinxstylestrong{Age is Older/Equal/Younger than} \sphinxhyphen{} Select one option among those listed.

\sphinxstylestrong{Marine Isotopic Stage} \sphinxhyphen{} Select a MIS from the list. Add new MIS definitions if necessary. If the MIS is already present in the list but a different definition is needed, use the text box below (Comments/details on MIS designation).

\sphinxstylestrong{Comments/details on MIS designation} \sphinxhyphen{} Provide comments or details on the designation above.

\sphinxstylestrong{Upper age (ka)} \sphinxhyphen{} Define the upper (older) age assigned to this record.

\sphinxstylestrong{Lower age (ka)} \sphinxhyphen{} Define the lower (younger) age assigned to this record.

\sphinxstylestrong{Notes on age determination} \sphinxhyphen{} Further notes on the age determination.

\sphinxstylestrong{Has duration estimated in years?} \sphinxhyphen{} Select “Yes” if a duration has been estimated for this entry.

\begin{sphinxadmonition}{note}{Note:}
The following fields appear if “Yes” is selected above
\end{sphinxadmonition}

\sphinxstylestrong{Duration (yrs)} / \sphinxstylestrong{Duration uncertainty (yrs)}

\sphinxstylestrong{Is a subzone of a parent Chronostratigraphy entry?} \sphinxhyphen{} Select “Yes” if a parent entry is available for this record.

\sphinxstylestrong{Parent record}


\chapter{Other age constraints}
\label{\detokenize{Other_age:other-age-constraints}}\label{\detokenize{Other_age::doc}}

\section{Description}
\label{\detokenize{Other_age:description}}
\sphinxstylestrong{Short Name} \sphinxhyphen{} Short name of the age constraint. This name will be selectable in other forms.

\sphinxstylestrong{Dating Technique} \sphinxhyphen{} Provide more details on the dating technique if needed.

\sphinxstylestrong{Description} \sphinxhyphen{} Add details on the record as needed.

\sphinxstylestrong{Reference(s)} \sphinxhyphen{} Select references.


\section{Age}
\label{\detokenize{Other_age:age}}
\sphinxstylestrong{Age is Older/Equal/Younger than} \sphinxhyphen{} Select one option among those listed.

\sphinxstylestrong{Marine Isotopic Stage} \sphinxhyphen{} Select a MIS from the list. Add new MIS definitions if necessary. If the MIS is already present in the list but a different definition is needed, use the text box below (Comments/details on MIS designation).

\sphinxstylestrong{Comments/details on MIS designation} \sphinxhyphen{} Provide comments or details on the designation above.

\sphinxstylestrong{Upper age (ka)} \sphinxhyphen{} Define the upper (older) age assigned to this record.

\sphinxstylestrong{Lower age (ka)} \sphinxhyphen{} Define the lower (younger) age assigned to this record.


\chapter{Indices and tables}
\label{\detokenize{index:indices-and-tables}}\begin{itemize}
\item {} 
\DUrole{xref,std,std-ref}{search}

\end{itemize}



\renewcommand{\indexname}{Index}
\printindex
\end{document}